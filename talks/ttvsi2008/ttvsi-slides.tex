\documentclass[compress,dvips,color=usenames,xcolor=dvipsnames]{beamer}

\usepackage{pstricks}
\usepackage{pst-tree}
\usepackage{pst-node}
\usepackage{proof}
\usepackage{alltt}


%% Comment this out to get CM sans-serif
\usepackage{euler} % like default CM, but with different, nice maths font

%%
% Use this (Boadilla) if you don't like the right sidebar
\usetheme{Boadilla}
%\usetheme{Goettingen} %nice toc sidebar on right
%%
% Use default for ultra-minimalist look
%\usetheme{default}

\useinnertheme{circles}
\usefonttheme[onlymath]{serif}
\setbeamertemplate{navigation symbols}{}

\newcommand{\ints}{\ensuremath{\mathbb{Z}}}
\newcommand{\nats}{\ensuremath{\mathbb{N}}}
\newcommand{\reals}{\ensuremath{\mathbb{R}}}
\newcommand{\rats}{\ensuremath{\mathbb{Q}}}
\newcommand{\bools}{\ensuremath{\mathbb{B}}}
\newcommand{\cpp}{\mbox{C\hspace{-.1em}+\hspace{-.05em}+}}

\newcommand{\strong}[1]{{\structcol #1}}
\newcommand{\stronger}[1]{\emph{\structcol #1}}
\newcommand{\strongest}[1]{\textbf{\structcol #1}}

\title{Defining a \cpp{} Semantics}
\subtitle{Topics in Operational Confusion}
\author{Michael Norrish}
\institute[NICTA]{Canberra Research Lab., NICTA}
\titlegraphic{\includegraphics[width=0.2\textwidth]{Nicta_vert.eps}}
\date{March 2008}

\definecolor{nblue}{rgb}{0.0,0.310,0.490}
\newrgbcolor{nictablue}{0.0 0.310 0.490}
\setbeamercolor{math text}{fg=nictablue}
\newcommand{\structcol}{\usebeamercolor[fg]{structure}}

\begin{document}
\frame{\titlepage}
\section{Introduction}
\begin{frame}{Outline}\tableofcontents\end{frame}

\begin{frame}{Why? What? How?}
In 2005, UK company QinetiQ approached me to write a semantics
  for \cpp{}

\bigskip
Decided early on to avoid static semantics\\[1mm]
  \quad\quad\dots {\footnotesize which lets me ignore operator and function overloading}

\bigskip
Aim to build on my PhD's C semantics
\end{frame}

\begin{frame}{Result}

A deeper appreciation of \cpp's features.

\bigskip
Some 12\hspace{.1em}000 lines of HOL4 source code (mainly definitions).

\bigskip
A report (120pp) describing the above in English.

\bigskip
Legal back-and-forth squeezed time available: work was done in 9
increasingly intense months.

\end{frame}

\begin{frame}{The Good, the Bad and the Ugly}

A Conclusion in Advance:
\begin{itemize}
\item I produced a good picture of the dynamics of real \cpp
\item I didn't have time to fill in anything like all the details
\item The community wants to make \cpp{} even more complicated
\end{itemize}

\end{frame}

\section{The Structure of an Operational Semantics}
\begin{frame}{\cpp{} Builds on C}

\cpp{} inherits a great deal from its ancestor C.

\bigskip
The basic semantics of expression evaluation, and interactions with
memory are taken from C essentially unchanged.

\bigskip
Control-flow is complicated by constructors, destructors and
exceptions, but there are no new statement forms in \cpp.

\bigskip My existing C semantics is a reasonable starting point.

\bigskip \footnotesize
Bonus:\\
\quad my 1998 HOL source upgrades to work with the HOL4 of 2007!

\end{frame}

\begin{frame}{Original C Semantics---Expressions}

Annoying Fact \#1:
\begin{quote}
   The evaluation of sub-expressions can happen in arbitrary order,
   possibly interleaving across sub-expressions.
\end{quote}

\bigskip
The natural expression of this are the small-step rules
\[
\infer{
  \langle e_1 \oplus e_2, \sigma_0\rangle \rightarrow_e
  \langle e_1' \oplus e_2, \sigma\rangle
}{
  \langle e_1, \sigma_0\rangle \rightarrow_e \langle e_1',\sigma\rangle
} \qquad
\infer{
  \langle e_1 \oplus e_2, \sigma_0\rangle \rightarrow_e
  \langle e_1 \oplus e_2', \sigma\rangle
}{
  \langle e_2, \sigma_0\rangle \rightarrow_e \langle e_2',\sigma\rangle
}
\]

\bigskip Apart from the fact that there is a shared state being
updated ($\sigma_0$ changes to $\sigma$), these are just like rules
for parallel reduction in CCS or the $\pi$-calculus.
\end{frame}

\begin{frame}{Original C Semantics---Statements}
  In 1995, it seemed natural to model statements in a big-step
  style.\\
  In 1995, I was young and na\"\i{}ve.

\bigskip
The big-step style makes handling scopes easy (the semantics itself
provides you with a free stack):
\[
\infer{
  \langle \textsf{Block}(s), \sigma_0\rangle
  \;\Downarrow_s\;
  \sigma[\textsf{vars} := \sigma_0.\textsf{vars}]
}{
  \langle s, \sigma_0\rangle \;\Downarrow_s\; \sigma
}
\]
where the states ($\sigma_0$, $\sigma$) contain a $\textsf{vars}$
component recording addresses and type information for variables in
scope.

\end{frame}

\begin{frame}{Original C Semantics---Function Calls}

  In a big-step style, function calls (which are expressions)
  naturally become ``atomic'':
  \[
  \infer{
    \langle \textsf{Fncall}(f,\vec{x}), \sigma_0 \rangle \rightarrow_e
    \langle \textsf{result}(\sigma),
    \sigma[\textsf{vars} := \sigma_0.\textsf{vars}]
    \rangle
  }{
    \langle \textsf{body}(f), \sigma'\rangle \;\Downarrow_s\; \sigma
  }
\]
where $\sigma'$ is like $\sigma_0$ but has extra variables
corresponding to $f$'s formal parameters bound to values from
$\vec{x}$.

\bigskip
``Atomic'' function calls are probably what C requires, and are
\strong{definitely} what \cpp{} requires.


\end{frame}

\begin{frame}{Original C Semantics---Statement Uglinesses}

  Expressions embedded inside statements require a call to the
  reflexive, transitive closure of the $\rightarrow_e$ relation:

\[
\infer{
  \langle\textsf{while}\;g \;\textsf{do}\;s,\sigma_0\rangle
  \;\Downarrow_s\;
  \sigma
}{
  \langle g, \sigma_0 \rangle \rightarrow_e^* \langle 0, \sigma\rangle
}
\]
Ugh!

\bigskip It is also impossible to model programs (like servers) that
deliberately loop.

\end{frame}

\begin{frame}{Making Statements Small-step}

If the rule-system doesn't give you a stack for free, you have to
model it explicitly:
\[
\infer{
  \langle \textsf{Block}(s), \sigma_0 \rangle \rightarrow_s
  \langle s, \sigma\rangle
}{
  \sigma =
  \sigma_0[\textsf{stk} := \sigma_0.\textsf{vars} :: \sigma_0.\textsf{stk}]
}
\]
\onslide<2->
But wait!  How will we know when to pop the stack?

\onslide<3->
\bigskip
One traditional approach to small-step statements is to use a
$\textsf{Skip}$ statement to mark the successful end of an
evaluation.

\bigskip But if $s$ is evaluated all the way to $\textsf{Skip}$,
how will we then know that there was a requirement to pop?

\end{frame}

\begin{frame}{Explicit Stacks}
Roughly the way I ended up doing it:
\[
\begin{array}{c}
\infer{
  \langle \textsf{Block}_\bot(s), \sigma_0 \rangle \rightarrow_s
  \langle \textsf{Block}_\top(s), \sigma\rangle
}{
  \sigma =
  \sigma_0[\textsf{stk} := \sigma_0.\textsf{vars} :: \sigma_0.\textsf{stk}]
}\\[5mm]
\infer{
  \langle \textsf{Block}_\top(s_0), \sigma_0\rangle \rightarrow_s
  \langle \textsf{Block}_\top(s), \sigma\rangle
}{
  \langle s_0,\sigma_0 \rangle \rightarrow_s \langle s, \sigma\rangle
}\\[5mm]
\infer{
  \langle\textsf{Block}_\top(\textsf{Skip}), \sigma\rangle
  \rightarrow_s
  \langle
     \textsf{Skip},
     \sigma[\textsf{stk} := t, \textsf{vars} := h]
  \rangle
}{\sigma.\textsf{stk} = h :: t}
\end{array}
\]

(Have to assume that all $\textsf{Block}$ forms are initially
$\textsf{Block}_\bot$.)

\bigskip
But what of ``atomic'' functions?
\end{frame}

\begin{frame}{Atomic Function Calls}

Scenario to avoid is having the expression
\[
f(x) + g(y)
\]
take a step inside $f$ followed by a step inside $g$.

\end{frame}

\begin{frame}{Atomic Function Calls with Continuations}

Drop the $s$ and $e$ subscripts on $\rightarrow$.

\medskip
Instead, have one arrow, and tags on the syntax $\textsf{St}$ and
$\textsf{Ex}$.

\medskip
Function call rule:
\[
\infer{
  \langle \textsf{Ex}(\textsf{FnCall}(f,\vec{x})), \sigma_0\rangle
  \rightarrow
  \langle \textsf{St}(\textsf{body}(f), (\lambda x. x)), \sigma'\rangle
}{
  \mbox{$\sigma'$ has $f$'s parameters installed etc}
}
\]

\medskip
A standard rule for sub-expression evaluation:
\[
\infer{
  \langle \textsf{Ex}(e_1 + e_2), \sigma_0 \rangle
  \rightarrow
  \langle \textsf{Ex}(e_1' + e_2), \sigma \rangle
}{
  \langle \textsf{Ex}(e_1), \sigma_0\rangle
  \rightarrow
  \langle \textsf{Ex}(e_1'), \sigma \rangle
}
\]


\end{frame}

\begin{frame}{Atomic Function Calls with Continuations}
If a sub-expression involved a function call:
\[
\infer{
  \langle \textsf{Ex}(e_1 + e_2), \sigma_0 \rangle
  \rightarrow
  \langle \textsf{St}(s, (\lambda x.\, c(x) + e_2)), \sigma\rangle
}{
  \langle \textsf{Ex}(e_1), \sigma_0 \rangle
  \rightarrow
  \langle \textsf{St}(s, c), \sigma \rangle
}
\]
$c$ is also a ``context''.  $(\lambda x.\,c(x) + e_2)$ is a function
on syntax.

\bigskip
When a $\textsf{St}(s,c)$ form becomes
$\textsf{St}(\texttt{return}(v),c)$, can pass the return value $v$ to
the containing expression by moving to $c(v)$.

\end{frame}

\begin{frame}{Teetering on the Knife-edge of Madness}
Getting even this far makes the whole enterprise feel disastrously
complicated.

\bigskip
Perhaps a ``pure continuation'' system is possible.\\
{\structcol
\strongest{Slogan:}
\parbox[t]{0.7\textwidth}{\itshape replace recursion within syntax with
  a ``list'' of steps still to perform.}}

\bigskip
For example,
\[
\infer{
  \langle \textsf{St}(\textsf{Block}(s), c), \sigma_0\rangle
  \rightarrow
  \langle \textsf{St}(s \;\textbf{;}\;\textsf{Pop}, c), \sigma\rangle
}{
  \sigma = \sigma_0[\textsf{stk} :=
       \sigma_0.\textsf{vars} :: \sigma_0.\textsf{stk}]
}
\]
($\textsf{Pop}$ is a new piece of ``syntax'' invented
specially)

\bigskip
\strongest{Against:}
\begin{itemize}
\item Handling interleaving of sub-expression evaluation looks
  complicated
\item The semantics grows ever harder for the reader to appreciate
\end{itemize}

\end{frame}

\section{Namespaces}

\begin{frame}{Variable-tracking the Young and Na\"\i{}ve Way}

This works for C:
  \begin{itemize}
  \item Maintain \strong{global} and \strong{current} maps for
    variables
    \begin{itemize}
    \item these are maps from names to types and addresses
    \end{itemize}
  \item When entering a function, initialise \strong{current} with
    \strong{global} map
  \item When a variable is declared, update \strong{current}.
  \end{itemize}

\end{frame}

\section{Building on the Shoulders of Giants}
\subsection{Multiple Inheritance}
\subsection{Templates}

\section{Proofs? Theorems?  Validation!?}


\section{Exceptions}


\end{document}