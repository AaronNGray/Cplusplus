\documentclass{llncs}
\usepackage{charter}
\usepackage{url}
\usepackage{alltt}

\newcommand{\cpp}{\mbox{C\hspace{-.3mm}+\hspace{-.8mm}+}}

\pagestyle{empty}
\title{Defining a \cpp{} Semantics}
\author{Michael Norrish}
\institute{Canberra Research Lab., NICTA}
\date{}

\begin{document}
\maketitle

\paragraph{Introduction}
In 2005, I was approached by QinetiQ to write a semantics for \cpp{}
in the style of the C semantics I wrote for my PhD~\cite{Norrish98}.
After some discussion, we decided that I could concentrate on the
dynamic semantics, and that I might ignore as much of the
(complicated) static semantics as I liked.  In particular, I decided
that this allowed me to ignore the complicated rules for operator and
function overloading.  This overloading is entirely resolved at
compile-time and might in this way be thought of as a static matter. 

I did commit to describing multiple inheritance, exceptions, templates
(for all that these are also a ``compile-time'' feature), object
lifetimes and namespaces.  This abstract highlights some of the issues
encountered in performing this task.  The result of the commission was
a document summarising the semantics, and some
\mbox{12\hspace{.8mm}000} lines of HOL4 sources.

\paragraph{From Na\"\i{}vety to Namespaces}
One can divide C's variables into two sorts: stack-allocated locals
and globals of permanent lifetime.  (A variable's visibility is an
orthogonal issue.)  Dynamically accessing a variable is simple: it
must either be a local, necessarily masking earlier locals and
globals, or it is a global.  Because of this, a C semantics can
maintain two maps for variables, its globals and its current
variables.  When a function is entered, one sets the current map to be
the global map, and then subsequently updates this with any further
local variables (including function parameters) that are encountered,
masking any other variables of the same name.

In the presence of namespaces (or classes), this simplistic scheme
falls apart.  Consider Figure~\ref{fig:nspaces}: on entering function
\texttt{ns1::f}, one cannot simply use the names belonging to
namespace \texttt{ns1}.  Done incorrectly, this will change the
\texttt{x} reference in \texttt{f} to point to the wrong variable.
When one includes the strange things that happen to names in the
presence of class declarations, the underlying semantics becomes even
more complicated.  Eventually, this was modelled with an explicit
``name resolution'' phase.
\begin{figure}
\begin{center}
\begin{minipage}{\textwidth}
\begin{alltt}
   int x = 3;                              val x = ref 3
   namespace ns1 \{                         structure ns1 = struct
     int f(int n) \{ return n + x; \}          fun f n = n + !x
     int x = 2;                              val x = ref 2
   \}                                       end
\end{alltt}
\end{minipage}
\end{center}
\caption{\cpp{} program (and SML equivalent) illustrating name
  resolution. }
\label{fig:nspaces}
\end{figure}


\paragraph{Non-Interleaving Function Calls in a Small-step Style} The
\cpp{} Standard~\cite{cpp-standard-iso14882} (following C) insists on
peculiar rules for the order of evaluation for expressions.  These
effectively require that expression evaluation be defined in a
small-step style.  Unfortunately, at the same time, the standard
insists that function calls do not interleave, which is easiest to
capture in a big-step semantics.  The earlier C semantics
in~\cite{Norrish98} merged these styles in an ugly way.  The new
\cpp{} approach uses a mixture of continuations and traditional
recursion on the syntactic structure of expressions.  This isn't
without its uglinesses; it's possible that a purely continuation-based
approach would be cleaner.  Unfortunately, continuations can
themselves introduce significant clutter, as well as some
incomprehensibility.

\paragraph{Validation and Proofs: the Ghost at My Banquet} The major
problem with a semantics as large as this one is that it is hard to
judge its correctness.  Proofs of type soundness and progression are
often used to provide some form of validation, but in the absence of a
good mechanisation of the type system, this option was not available
with this semantics.  In the case of \cpp, the more important question
is whether or not the semantics corresponds to the ``real'' definition
of the language.  There are no deep proofs in the semantics, but there
are some ``sanity'' theorems of various forms.

\paragraph{Other Features} Two significant pieces of existing work
enabled important parts of the semantics to be taken ``off the
shelf''.  In particular, the treatment of multiple inheritance
followed that given in the paper by Wasserrab \emph{et
  al}~\cite{wasserrab-nst-OOPSLA06}. The only significant adjustment
required when incorporating this material was to remove the Java-style
assumption that ``everything is a pointer to something on the heap''.
Capturing the layout of objects with \texttt{virtual} ancestors was
particularly involved.

Secondly, the semantics also drew on the description of templates in
Siek and Taha~\cite{DBLP:conf/ecoop/SiekT06}.  \cpp{} templates are
one of the language's most complicated features, and are modelled as a
separate phase that happens before any dynamic behaviour occurs.

\begin{thebibliography}{1}

\bibitem{cpp-standard-iso14882}
{\em Programming Languages---C++}, 2003.
\newblock ISO/IEC 14882:2003(E).

\bibitem{Norrish98}
Michael Norrish.
\newblock {\em {C} formalised in {HOL}}.
\newblock PhD thesis, Computer Laboratory, University of Cambridge, 1998.
\newblock Also available as Technical Report 453 from \url{ht
  tp://www.cl.cam.ac.uk/TechReports/UCAM-CL-TR-453.pdf}.

\bibitem{DBLP:conf/ecoop/SiekT06}
Jeremy Siek and Walid Taha.
\newblock A semantic analysis of \cpp{} templates.
\newblock In Dave Thomas, editor, {\em ECOOP}, volume 4067 of {\em Lecture
  Notes in Computer Science}, pages 304--327. Springer, 2006.

\bibitem{wasserrab-nst-OOPSLA06}
Daniel Wasserrab, Tobias Nipkow, Gregor Snelting, and Frank Tip.
\newblock An operational semantics and type safety proof for multiple
  inheritance in \cpp.
\newblock In {\em {OOPSLA}~'06: Object oriented programming, systems,
  languages, and applications}. ACM Press, 2006.
\newblock Available from
  \url{http://isabelle.in.tum.de/~nipkow/pubs/oopsla06.html}.

\end{thebibliography}

\end{document}



