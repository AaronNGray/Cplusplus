\documentclass[dvips,xcolor=pst,color=usenames]{beamer}

\usetheme{Madrid}
\usepackage{url}
\usepackage{pstricks}
\usepackage{pst-tree}
\usepackage{pst-node}
\usepackage{proof}
%\usepackage{pst-grad}
\usepackage{fancyvrb}

\setbeamertemplate{navigation symbols}{}
\setbeamercolor{math text}{fg=blue}
\newcommand{\structcol}{\usebeamercolor[fg]{structure}}

\title[Formally Defining C++]{Formally Defining C++\\
  Why, How, \& What Exactly}
\author{Michael Norrish}
\institute{National ICT Australia}
\titlegraphic{\includegraphics[width=0.3\textwidth]{nicta-colour-on-pale.eps}}
\date{25 April 2006}

\begin{document}

\frame{\titlepage}

\section*{Introduction}

\frame{
  \frametitle{Outline}
  \tableofcontents
}

\section{Why?}
\subsection{Why Formal Language Description}
\frame{
  \frametitle{Why Describe a Language?}

\textbf{Reason \#1:}

\begin{quote}
To establish a contract between the {\red users} and
{\red implementors} of a language
\end{quote}

\pause
\begin{center}
\begin{tabular}{lp{0.6\textwidth}}
{\structcol Users} & Programmers; those who submit source code to a language
  implementation and expect a particular behaviour in return\\[3mm]
{\structcol Implementors} & Programmers; those who write a language
  implementation which other programmers will feed with source code
\end{tabular}
\end{center}

}

\frame{
  \frametitle{A Language Contract}

{\structcol \bfseries Manages Expectations}

\begin{itemize}
\item Users expect \texttt{\structcol i = i + 1} to add one to
  variable \texttt{\structcol i}

\item Implementors can rely on users not attempting to divide by
  zero (in C/C++ anyway)
\end{itemize}

\pause
\bigskip\bigskip
{\structcol \bfseries Is Not an Implementation}

\begin{itemize}
\item Even for a language like Perl.
  \begin{itemize}
  \item After all, people report bugs
    against Perl implementations.   Something other than the
    implementation is defining what is and is not correct.
  \end{itemize}
\end{itemize}

}

\frame{
  \frametitle{Language Contracts are Written in English}

To wit (ISO/IEC 14882:2003, \S3.5, para 8):
\begin{quote}
\small
Names not covered by these rules have no linkage.  Moreover, except as
noted, a name declared in a local scope (3.3.2) has no linkage.  A
name with no linkage (notably, the name of a class or enumeration
declared in a local scope (3.3.2)) shall not be used to declare an
entity with linkage.  If a declaration uses a typedef name, it is the
linkage of the type name to which the typedef refers that is
considered. [Example: ...] This implies that names with no linkage
cannot be used as template arguments (14.3).
\end{quote}

If the user and the implementor disagree on the meaning of the
contract, how can such a dispute be resolved?

}

\frame{
  \frametitle{Formal Contracts}

  [``Formal'' $=$ ``Mathematical'']

  \medskip
  \begin{itemize}
  \item Mathematical reasoning is a route to certainty
  \item Computers are designed to behave like (discrete-state)
    mathematical objects \pause
    \begin{itemize}
    \item when they don't, the ``hardware contract'' has been broken
    \end{itemize}
    \pause
    \medskip
  \item Programming languages can be described mathematically
    \medskip
    \pause
  \item More, high-level programming languages can be described
    \emph{abstractly}, in a high-level way
    \begin{itemize}
    \item without requiring the description of a translation to machine
      code
    \item will describe how this is done when I describe {\structcol ``How?''}
    \end{itemize}
  \end{itemize}

}

\frame{
  \frametitle{Advantages of Formal Contracts}

  Mathematical methods can be used to determine exactly what a
  language guarantees/requires.

  \bigskip\bigskip
  A rigorous understanding of a language can motivate definition of
  {\red language subsets}.  (E.g., MISRA C)

  \bigskip A formal definition for a language can be the basis for the
  development of powerful {\red static analysis tools}.
}

\frame{
  \frametitle{Static Analysis Tools}

  \begin{itemize}
  \item Many successful tools exist in this domain already, but most
    are
    \begin{itemize}
    \item unsound,
    \item incomplete, and
    \item can only be used to find bugs, not to
      provide much certainty about correctness
    \end{itemize}

    \bigskip
  \item Full program verification: the ultimate static analysis.

    But
    \begin{itemize}
    \item retrospective program verification is impossible
    \item reliability must be built in from the outset
    \item formality is needed in the specification as well as the
      description of the implementation language
    \end{itemize}

  \end{itemize}

}

\frame{
  \frametitle{One Possible Static Analysis Scenario}

\psset{unit=1mm,linewidth=.1mm}
\begin{pspicture}(120,70)
\psframe[fillstyle=solid,fillcolor=yellow!15!white,linestyle=none](-2,71)(122,0)
% high-level source
\rput(0,-2.5){
  \psframe[framearc=.2,linecolor=blue,linewidth=.2mm,fillstyle=solid,fillcolor=white](0,70)(40,35)
  \rput(20,63){\parbox{3cm}{\centering \blue High-level Source\\(Spec)}}
  \psset{linecolor=blue,linestyle=dotted,linewidth=.2mm}
  \rput(0,-5){
    \psline(5,60)(35,60)
    \psline(5,55)(35,55)
    \psline(5,50)(35,50)
    \psline(5,45)(35,45)
  }
}
% low-level code
\rput(0,0){
  \psframe[shadow=true,framearc=0,linecolor=Brown](80,70)(120,30)
  \rput(100,65){\textcolor{Brown}{\texttt{Low-level Code}}}
  \psset{linecolor=Brown,linestyle=dotted,linewidth=.2mm}
  \rput(80,-5){
    \psline(5,60)(35,60)
    \psline(5,55)(35,55)
    \psline(5,50)(35,50)
    \psline(5,45)(35,45)
    \psline(5,40)(35,40)
  }
}
% down left
\pscurve[linecolor=blue,linewidth=1.25]{->}(20,32.5)(25,22)(50,18)(55,15)
% down right
\pscurve[linecolor=Brown,linewidth=1.25]{->}(100,30)(95,22)(70,18)(65,15)
% translation arrow
\rput(0,0){
  \psline[linewidth=.8]{->}(40,50)(80,50)
  \rput(60,55){\psovalbox[fillcolor=CornflowerBlue,fillstyle=solid,linestyle=none]{\textbf{Translation}}}
}
\rput(60,10){\parbox{4cm}{\centering Verification\\(Equivalence Check)}}
\end{pspicture}

The translation might be done by a third-party tool, but rigorous
descriptions of source and target languages are required.

}


\subsection{Why C++}
\frame{
  \frametitle{Why C++ ?}

C++
\begin{itemize}
\item is a popular object-oriented application-level language
\item provides higher-level abstractions than C
\item \dots while still compiling to efficient code
\item high-level abstractions can lead to cleaner/safer/more reliable implementations
\end{itemize}

}

\frame{
  \frametitle{C++ Features}

  \begin{itemize}
  \item \textbf{Inheritance} (multiple, public and private), giving
    \begin{itemize}
    \item \textbf{Sub-class polymorphism} (cats behave like mammals,
      like animals)
    \end{itemize}
  \item \textbf{Parametric Polymorphism} (templates, ``generic code'')
  \item \textbf{Function Overloading} (functions of the same name with
    different types)
  \item \textbf{Exceptions}
  \item \textbf{Hierarchical Name-spaces}
  \item \textbf{Tweaks/Additions to C} (references (not just
    pointers), extended sorts of casts, the \texttt{\structcol const}
    modifier)
  \end{itemize}


}

\frame{
  \frametitle{Why Not Formalise C++ ?}

(See previous slide\dots)

\bigskip
Semantic complexity of many interacting features is intimidating.
\begin{itemize}
\item ``No-one knows all of C++''
\end{itemize}

\bigskip
How would one know that a formal semantics was correct?
\begin{itemize}
\item Indeed, what does \emph{correctness} mean here?
\end{itemize}

}

\section{How?}

\frame{
  \frametitle{How?}
\tableofcontents[current]
}
\subsection{Operational Semantics---Simple Examples}
\frame{
  \frametitle{Operational Semantics}

I propose to write a formal description of C++ using the technique
known as operational semantics.

\bigskip This can been seen as
\begin{quote}
\structcol writing an interpreter in logic
\end{quote}

\pause\bigskip This is an interpreter that can use the full (not
necessarily executable) expressive-ness of logic and mathematics
(sets, quantifiers, functions \dots).

\medskip It can
\begin{itemize}
\item model under-specification
\item model non-determinism
\end{itemize}

}

\newcommand{\semarrow}{\;\,\rightarrow\;\,}



\frame{
  \frametitle{Operational Semantics---Generalities}

An operational definition typically is a set of rules of the form
\[
\infer{\mathit{start} \semarrow \mathit{result}}{
    \mathit{premiss}_1 &
    \cdots &
    \mathit{premiss}_n}
\]
meaning
\begin{quote}
\structcol
If the premisses hold then it is the case that $\mathit{start}$ can
become/evolve~to/evaluate~to $\mathit{result}$.
\end{quote}

\bigskip
(Use of {\structcol can} rather than {\structcol must} above allows
for non-determinism.)
}

\frame{
  \frametitle{Operational Semantics---Example Rules}

Rules for $\textsf{if}$:
\[
\begin{array}{c}
\infer{\textsf{if}\;G\;\textsf{then}\;e_1\;\textsf{else}\;e_2
  \semarrow v_1}{
  G \semarrow \textsf{True} & &
  e_1 \semarrow v_1
}\\[1cm]
\infer{\textsf{if}\;G\;\textsf{then}\;e_1\;\textsf{else}\;e_2
  \semarrow v_2}{
  G \semarrow\textsf{False} & &
  e_2 \semarrow v_2
}
\end{array}
\]

These are fine for $\textsf{if}$-expressions that do not have side
effects.

}

\frame{
  \frametitle{Operational Semantics---Modularity}

  The ``style'' is to write independent rules for specific syntactic
  forms.

  \bigskip
  For example, take the previous slide's rules for $\textsf{if}$, and
  add in a rule for $+$:
\[
\infer{e_1 \,{\red +} \,e_2 \semarrow v_1 + v_2}{
  e_1 \semarrow v_1 & & e_2 \semarrow v_2
}
\] (Note two different flavours of $+$ !)

\bigskip
Adding in treatment of language's arithmetic operators is simple.

}

\frame{
  \frametitle{Operational Semantics---Rules Can Refer to Each Other}


With rule for unary negation:
\[
\infer{{\red -}e \semarrow -v}{e \semarrow v}
\]

\bigskip
can define binary subtraction in terms of $+$:
\[
\infer{e_1\,{\red -}\,e_2 \semarrow v}{e_1 \,{\red+\,-}e_2 \semarrow
  v}
\]

\bigskip
(Rules need not be \emph{primitive recursive}: premiss in rule for
binary subtraction is ``larger'' than conclusion.)

}

%% slides on extensibility - handling guards that evaluate to numbers

\frame{
  \frametitle{Operational Semantics---Extensibility}

Rules for $\textsf{if}$
\[
\begin{array}{c}
\infer{\textsf{if}\;G\;\textsf{then}\;e_1\;\textsf{else}\;e_2
  \semarrow v_1}{
  G \semarrow \textsf{True} & &
  e_1 \semarrow v_1
}\\[1cm]
\infer{\textsf{if}\;G\;\textsf{then}\;e_1\;\textsf{else}\;e_2
  \semarrow v_2}{
  G \semarrow\textsf{False} & &
  e_2 \semarrow v_2
}
\end{array}
\]
assume that $G$ will evaluate to $\textsf{True}$ or $\textsf{False}$.

\medskip
What if we want to allow a C-like situation where non-zero numbers
count as \textsf{True}?

}

\frame{
  \frametitle{Operational Semantics---Extensibility}

To handle numbers as truth values, add two new rules:
\[
\begin{array}{c}
\infer{\textsf{if}\;G\;\textsf{then}\;e_1\;\textsf{else}\;e_2
  \semarrow v_1}{
  G \semarrow n & n \neq 0&
  e_1 \semarrow v_1
}\\[1cm]
\infer{\textsf{if}\;G\;\textsf{then}\;e_1\;\textsf{else}\;e_2
  \semarrow v_2}{
  G \semarrow 0 & &
  e_2 \semarrow v_2
}
\end{array}
\]

Before there was no behaviour defined for guards that evaluated to
numbers; now our system has been extended to allow it.

\medskip
(Could equally cause such an event to raise an exception/abort\dots)

\pause\medskip
Note also how easy it would be to get the system to evaluate (and
ignore) the branch not taken.

}


%% slides on adding environments/states for side effects
\frame{
  \frametitle{Operational Semantics: Adding More Complexity}

Programming languages usually include variables.

\medskip
Variables have values that need to be looked up in some sort of
{\red environment} or {\red state}.

\medskip
Change pattern of rule to be
\[
\infer{\langle\mathit{thing},\sigma_0\rangle \semarrow
  \mathit{result}}{\cdots}
\]
where the $\mathit{result}$ will probably include a $\sigma$,
representing the updated state of memory.

}

\newcommand{\bfsemi}{\textsf{\bfseries ;}}
\frame{
  \frametitle{Operational Semantics: Statements}

The $\mathit{result}$ is just a state.

\bigskip
Sequential composition:
\[
\infer{\langle \mathit{stmt}_1\,\bfsemi\,\mathit{stmt}_2,\;\sigma_0\rangle
  \semarrow \sigma_2}{
  \langle\mathit{stmt}_1,\;\sigma_0\rangle \semarrow \sigma_1 &
  \langle\mathit{stmt}_2,\;\sigma_1\rangle \semarrow \sigma_2}
\]
\bigskip
A loop rule:
\[
\infer{\langle\textsf{while}\;G\;\textsf{do}\;\mathit{body},\;\sigma_0\rangle
  \semarrow \sigma}{
  \langle\textsf{if}\;G\;\textsf{then}\;(\mathit{body}\bfsemi\;\textsf{while}\;G\;\textsf{do}\;\mathit{body}),\;\sigma_0\rangle \semarrow \sigma}
\]
}

\newcommand{\plusplus}{\!+\!\hspace{-1pt}+}
\frame{
  \frametitle{Operational Semantics: Expressions with Side Effects}

Form of rule becomes:
\[
\infer{\langle e,\sigma_0\rangle \semarrow
  \langle v, \sigma\rangle}{\cdots}
\]


\bigskip
For example, a post-increment operator \emph{\`a la} C:
\[
\infer{\langle \mathit{var}\plusplus, \;\sigma\rangle
  \semarrow \langle v,\;
  \sigma[\mathit{var}\mapsto v + 1]\rangle}{
  \mbox{state }\sigma\mbox{ maps }\mathit{var}\mbox{ to value }v}
\]

\bigskip\pause (Note though that this is {\red not} a valid way of
modelling what happens in C/C++.)

}

\frame{
  \frametitle{Operational Semantics: Types, Well-formedness}

A similar style can be used to specify types for expressions.

\medskip
Often written with extra typing context $\Gamma$ (which records
types for variables):
\[
\infer{\Gamma \vdash e : \tau}{\cdots}
\]
read: ``Given context $\Gamma$, expression $e$ has type $\tau$''

\pause\medskip
A colon instead of an arrow ($\rightarrow$); \\
\qquad a static description instead of a dynamic one.

\medskip
For example:
\[
\infer{\Gamma \vdash e\plusplus : \tau}{
  \Gamma \vdash e : \tau &
  e\mbox{ an l-value} &
  \tau\mbox{ a scalar type}
}
\]

}

\frame{
  \frametitle{Operational Semantics: Summary}

\begin{itemize}
\item a well-understood, standard, technology
\item extensible
\item modular
\item mechanisable (\emph{i.e.}, allowing mechanical proof)

\medskip
\item capable of dealing with real-world languages

\end{itemize}

}


\subsection{C: Witness to Practicality}

\frame{
  \frametitle{Specifying C}

Features of C demonstrates strengths of approach.

\medskip
Capable of handling C (and C++) exotica such as
\begin{itemize}
\item sequence points
\item unspecified order of evaluation
\item low-level memory model (pointer arithmetic, \&c)
\item under-specified (``implementation defined'') aspects such as
  \begin{itemize}
  \item number of bits in a byte
  \item size of types such as \texttt{\structcol long} and
    \texttt{\structcol short}
  \item representation of negative numbers (twos-complement,
    ones-complement, sign-magnitude)
  \end{itemize}
\end{itemize}

}

% handling interleaved execution order

\frame[containsverbatim]{
  \frametitle{C Exotica: Interleaved Execution Order}

In C, the expression
\begin{Verbatim}[formatcom=\structcol]
   printf("a") + printf("b") + printf("c")
\end{Verbatim}
can print any of the six orderings of \texttt{\structcol a}, \texttt{\structcol
  b}, \texttt{\structcol c}.

\bigskip
This can be modelled with rules for $+$:
\[
\begin{array}{c@{\hspace{1cm}}c}
\infer{e_1 \,{\red +}\, e_2 \semarrow e_1' \,{\red +}\, e_2}{e_1 \semarrow e_1'} &
\infer{e_1 \,{\red +}\, e_2 \semarrow e_1 \,{\red +}\, e_2'}{e_2 \semarrow e_2'}\\[5mm]
\multicolumn{2}{c}{
\infer{v_1 \,{\red +}\, v_2 \semarrow v_1 + v_2}{}}
\end{array}
\]

(This represents a slightly different style of rule than
previously---syntax evolves ``step-by-step''.)

}

\subsection{Probable Approaches}

\frame{
  \frametitle{Approaches to Handling C++ Features---Examples}

{\structcol \bfseries Object orientation}
\begin{itemize}
\item Classes are {\structcol struct} values with added ``member'' functions
\item Runtime sub-class inheritance is typically implemented with a
  \texttt{\structcol vtable} (a pointer in the \texttt{\structcol
    struct})
  \begin{itemize}
  \item a formal semantics can model this by encoding the value's type
    inside the \texttt{\structcol struct}
  \end{itemize}
\end{itemize}

{\structcol \bfseries References}
\begin{itemize}
\item Simply modelled as addresses
\end{itemize}

\smallskip
{\structcol \bfseries Templates}
\begin{itemize}
\item Less clear, particularly as all the ``action'' happens at
  compile time.
\end{itemize}

}

\subsection{Validation of the Semantics}

\frame{
  \frametitle{Validation}

It's impossible to \emph{\structcol prove} that a formal semantics
corresponds to the ISO standard (a prose document).

\bigskip Nor, with as complicated a language as C++, would simply
eye-balling the text of the formal description be enough.

\bigskip
How then can we accept that another description is correct?

\bigskip
(One might ask the same question of any C++ compiler\dots)


}

% done mechanically at all

\frame{
  \frametitle{Validation Through Mechanisation}

\begin{center}
\fbox{The formal description will be written with HOL}
\end{center}

\bigskip\bigskip
HOL is
\begin{itemize}
\item an interactive proof assistant (theorem proving) system
\item implements higher-order logic (thus the name)
\item well-established technology (20+ years' history)
\item well-suited to operational semantics
\end{itemize}

\bigskip
Using a tool means pen-and-paper errors (typos \&c) can't arise.

\smallskip
(Analogous to getting a compiler's source code to compile.)

}

% simple sanity-check proofs possible
\frame{
  \frametitle{Validation Through Proof}

\begin{center}
\fbox{Proofs can certify important properties of the description}
\end{center}

\bigskip
\bigskip
Properties are not intended to be exciting new results, but things
expected to be true---also known as ``sanity checks''.

\bigskip
Properties might include
\begin{itemize}
\item Equivalence of certain forms, e.g.:
\begin{eqnarray*}
\texttt{if (0) }s_1\texttt{ else }s_2 &=& s_2 \\
\texttt{for (i=0;i<10;i++);} &=& \texttt{i = 10;}
\end{eqnarray*}
\item Simple facts, e.g.:
\begin{quote}
if an expression does not contain an assignment, or any function
calls, it does not affect the contents of memory
\end{quote}
\end{itemize}


}

% simple symbolic evaluation of straight-line code
\frame{
  \frametitle{Validation Through Execution}

\begin{center}
\fbox{Semantics can be used as an interpreter}
\end{center}

\bigskip\bigskip
Just as compilers can be tested against validation suites, so too can
the formal description.

\bigskip
This is a specialised form of proof, about the behaviour of specific
programs.

}

\section{What Exactly?}

\frame{
  \frametitle{What Exactly?}

The semantics can cover more and more of C++ in phases.
\begin{center}
\begin{tabular}{lp{0.6\textwidth}}
C++$\black_0$ & C (existing work)\\[3mm]
C++$\black_1$ & add basic object orientation: dynamic dispatch,
inheritance and reference types\\ [3mm]
C++$\black_2$ & Multiple inheritance, \texttt{\structcol const}
keyword, object lifetimes \\[3mm]
C++$\black_3$ & Access modifiers (\texttt{\structcol private},
\texttt{\structcol public}), templates, exceptions\\[3mm]
\dots & \dots
\end{tabular}
\end{center}

}

\subsection{Vulnerability Analysis}
\frame{
  \frametitle{Analysis of Vulnerabilities}

A working practice document like MISRA C forbids programmers from
doing {\structcol ``bad things''}.

\bigskip
But there are multiple sorts of {\structcol ``bad thing''}.

\bigskip For example, the library function \texttt{\structcol strcpy}
is a classic source of buffer overflow bugs
\begin{itemize}
\item it's easy to check that a programmer hasn't broken the rule
  forbidding \texttt{\structcol strcpy}'s use when a program is
  written
\end{itemize}

\bigskip
But other restrictions are not so easy\dots

}

\frame{
  \frametitle{Dynamic Vulnerabilities}

More {\structcol ``bad things''}:
\begin{itemize}
\item Dividing by (integer) zero
\item Running off the end of an array
\end{itemize}

\bigskip
These can't be eliminated/avoided in a decidable way, except by
drastic measures.

\bigskip Nor do typical C/C++ run-time implementations detect these as
they occur (contrast Java).

}

\frame{
  \frametitle{Dynamic Vulnerabilities}

Dynamic vulnerabilities might be avoided by
\begin{itemize}
\item \textbf{testing}---``can only show the presence, not the
  absence, of errors''
  \begin{itemize}
  \item failure of run-time systems to trap errors can make bugs
    harder to detect
  \end{itemize}

\smallskip
\item \textbf{static analysis}---``argue that a particular run-time
  fault can not occur''

  \smallskip
  Where ``argument'' might be:
  \begin{itemize}
  \item code inspection
  \item tool use
  \item formal verification (must be built on a formal language
    description)
  \end{itemize}
\end{itemize}



}


\frame{
  \frametitle{Summary}




}


\end{document}



%%% Local Variables:
%%% mode: latex
%%% TeX-master: t
%%% End:
