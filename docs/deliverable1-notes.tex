\documentclass[11pt]{article}

\usepackage{charter}
\usepackage{alltt}
\usepackage{url}
\usepackage{proof}

\title{Notes on Deliverable 1 (July 2006)}
\author{Michael Norrish\\{\small \texttt{Michael.Norrish@nicta.com.au}}}
\date{}

\newcommand{\cpp}{\mbox{C\hspace{-0.5pt}+\hspace{-1.5pt}+}}

\begin{document}
\maketitle

\section{Form}

This deliverable consists of a compressed \texttt{tar}-file, that when
unpacked consists of a directory called \texttt{qinetiq-cpp}, which in
turn contains three directories
\begin{itemize}
\item \texttt{holsrcs}, containing the HOL source files of the
  mechanisation.  These files will build with the version of HOL4
  present in the CVS repository at SourceForge, with timestamp
  \texttt{2006-06-30 04:05Z}.  See Section~\ref{sec:getting-hol}
  below for instructions on how this version of HOL can be retrieved,
  and how the deliverable's HOL source files can then be built and
  checked.
\item \texttt{talks}, containing the \LaTeX{} source and a PDF for the
  talk presented at the DARP meeting in Newcastle in April 2006.  The
  source assumes that the \texttt{latex-beamer} and \texttt{PSTricks}
  packages are available.
\item \texttt{docs}, containing \LaTeX{} sources and a PDF version of
  this document.
\end{itemize}

\subsection{Building HOL Source-Files}
\label{sec:getting-hol}

\paragraph{Getting HOL From SourceForge}

To get a particular, dated, version of the HOL4 sources from the CVS
repository, one must first issue the command

{\small
\begin{verbatim}
   cvs -d:pserver:anonymous@hol.cvs.sourceforge.net:/cvsroot/hol login
\end{verbatim}
}

When prompted for a password, just press \texttt{Enter} to send a null
response.  The check-out of source code from SourceForge can now
proceed.  The source code fits into 60MB.  Issue the command

{\small
\begin{alltt}
   cvs \textit{repository-spec} co -D \textit{date} hol98
\end{alltt}
}

\noindent where \textit{\ttfamily repository-spec} is the string

{\small
\begin{alltt}
   -d:pserver:anonymous@hol.cvs.sourceforge.net:/cvsroot/hol
\end{alltt}
}

\noindent (also used in the \texttt{login} command), and where
\textit{\ttfamily date} is the desired date, best specified as an
ISO~8601 string enclosed in double-quotes.  For example,
\texttt{"2006-06-30 04:05Z"}.

Once a copy of the sources have been downloaded, further commands can
be used to update this copy to correspond to different dates.  As long
as such commands are issued from within the \texttt{hol98} directory,
the repository specification can be omitted.  The update command is

{\small
\begin{alltt}
   cvs update -d -D \textit{date}
\end{alltt}
}

\paragraph{Installing HOL} Once the sources have been downloaded, the
installation instructions from the page at
\url{http://hol.sourceforge.net} should be followed to build a copy of
HOL.  An installation of the Moscow~ML compiler (v2.01) will also be
required.

\paragraph{Building Deliverable Sources}
When HOL4 has been installed, the \texttt{Holmake} program (found in
the \texttt{hol98/bin} directory) can be run in the \texttt{holsrcs}
directory to create and check the logical theories.

\section{Content}

This deliverable delivers an updated HOL semantics for C, based on my
PhD and the accompanying source code.  The most significant change is
moving the dynamic semantics of statements (and declarations) to be in
a ``small step style''.  This improvement is one I believed necessary
even as I finished my thesis research.  There is some further
discussion of this change in Section~\ref{sec:small-step-stmts}
below.  There is a detailed manifest describing the files and their
inter-relationships in the file \texttt{holsrcs/MANIFEST}.

In addition, I have also updated the formalisation of memory, making
use of theories for fixed-width words that were not available in 1998.
This paragraph material occurs in the source file
\texttt{memoryScript.sml}. There is now a type that directly
corresponds to the standard's notion of \texttt{byte}.  At the same
time, there are now additional functions encoding the relationship
between sequences of bytes and values.  For integral types, there are
(partial) functions \texttt{REP\_INT} from integer values into byte
seqeunces, and \texttt{INT\_VAL} reversing this.  I have striven to
keep these funcations under-specified, in accordance with what the
standard requires.

In fact, as they stand, these functions are not specified enough.  In
particular, for the moment, I have only characterised the functions'
behaviour over values of character type.  This
``over-under-specification'' does not impede definition of semantics
elsewhere, but does represent work that still needs doing.

Finally, I have added a new treatment of external declarations.  This
\texttt{emeaning} constant in the \texttt{dynamics} theory specifies
how global variables are declared, and how functions are given their
definitions.  This is a necessary prelude to any treatment of
``translation units'' and their inter-linkage.

\subsection{Statements in a Small-step Style}
\label{sec:small-step-stmts}

One might imagine that restating the statement part of the dynamic
semantics in a small-step style should be easy.  The literature
contains many examples of how to express constructs such as
\texttt{while} and \texttt{if} in this style.  However, this it is not
as straightforward as one might think because of the need to prevent
function bodies from interleaving.

Imagine a program such as that in Figure~\ref{fig:two-functions}, and
how one might evaluate the return-expression in \texttt{main}.  If one
simply expanded the bodies of the called functions into the expression
as the functions were ready to be called, one would be permitting the
simultaneous evaluation of the bodies of \texttt{f} and \texttt{g}.
But the \cpp{} standard explicitly forbids this (\S1.8~fn8), and the C
standard also hints that it is forbidden.
\begin{figure}[htbp]
\begin{verbatim}
   int global;
   int f(int x) { return global * 2 + x; }
   int g(int y) { while (y > 0) { global += 2; y--; } }

   int main(void) {
     global = 10;
     return f(6) + g(10);
   }
\end{verbatim}
\caption{Where Functions Must Not Interleave}
\label{fig:two-functions}
\end{figure}

One has to arrange the semantics so that expression evaluation can
continue non-deterministically until a function call is encountered
and the function call is made (after arguments have been evaluated).
At this point, all further evolution of the program must occur within
the function body, no matter how deeply nested the function call may
have been within an enclosing expression.  (This problem did not occur
when statement evaluation was big-step because the hypothesis in the
expression rule for function call was the statement rule that called
for the complete evaluation of the function body.)

As before, there are superficially two reduction arrows in the
semantics: $\rightarrow_e$ for expression evaluation and
$\rightarrow_s$ for statement evaluation.  However, in reality there
is just one arrow, which is a binary relation on \emph{extended
  expressions} tupled with program states.  An extended expression is
either
\begin{itemize}
\item a syntactic expression coupled with a side effect information
  record (containing the three fields, \texttt{update\_map},
  \texttt{ref\_map} and \texttt{pending\_ses}, ($R$, $\Upsilon$ and
  $\Pi$ in the terminology of my thesis)); or
\item a statement coupled with a continuation, which latter is a
  function that takes a value and returns an expression.  This latter
  is used to recreate the expression in which the function call
  that generated the statement occurred.  Also, all expressions within
  statements (such as those that appear as guards in loops and
  \texttt{if}-statements), are actually extended expressions.
\end{itemize}
Most of the time reduction occurs between expressions and expressions,
or between statements and statements, allowing one to imagine that one
has $\rightarrow_e$ and $\rightarrow_s$ at play.

However, there are four rules that mix the types of extended
expression.  When a function call is made, we get a rule that is
roughly of the form
\[
\infer{
   \langle \textsf{Ex}(f(\mathit{args}), \mathit{se}), \sigma_0\rangle  \rightarrow
   \langle \textsf{St}(body(f), (\lambda v.v)), \sigma\rangle
}{\mathit{se} \mbox{ is empty of side effects} &
  \mbox{$f$ and $args$ are fully evaluated}}
\]
and where $\sigma$ is the same as $\sigma_0$ but with bindings
established between the values in $\mathit{args}$ and the formal
parameters in the function $f$.  The continuation accompanying the
expression at this point is simply the identity.  When the body
returns a value, that value will simply be substituted for what was
the function call, and expression evaluation can continue.  (The
\textsf{Ex} tag indicates that we are looking at an expression, while
\textsf{St} indicates that the value is a statement with its continuation.)

The next rule extends a continuation.  In the thesis semantics, we
have the rule
\[
\infer{\langle {\cal E}[e_0], \sigma_0\rangle \rightarrow_e
  \langle {\cal E}[e],\sigma\rangle}{
  \langle e_0,\sigma_0\rangle \rightarrow_e \langle e,\sigma\rangle}
\]
with $\cal E$ a context (a function from expressions to expressions).
In the new semantics we have
\[
\infer{\langle \textsf{Ex}({\cal E}[e_0], \mathit{se}_0), \sigma_0\rangle \rightarrow
  \langle \textsf{Ex}({\cal E}[e],\mathit{se}),\sigma\rangle}{
  \langle \textsf{Ex}(e_0,\mathit{se}_0),\sigma_0\rangle \rightarrow_e
  \langle \textsf{Ex}(e,\mathit{se}),\sigma\rangle}
\]
which is directly analogous.  But in addition, we also have
\[
\infer{\langle \textsf{Ex}({\cal E}[e_0],\mathit{se}_0),
  \sigma_0\rangle \rightarrow
  \langle \textsf{St}(s, {\cal E} \circ c), \sigma\rangle}{
  \langle \textsf{Ex}(e_0, \mathit{se}_0),\sigma_0\rangle
  \rightarrow
  \langle \textsf{St}(s, c), \sigma\rangle}
\]
which can be read as saying that if a nested expression evaluates to a
statement (i.e., a function call has begun), then the same function
call must be held to have begun at the outer level, but that the
continuation must insert the return value somewhat deeper into the
expression's syntax tree.  This is reflected by the (function)
composition of $\cal E$ with the continuation $c$ from the rule's
hypothesis.

These rules have specified what happens when an expression evaluation
switches to a statement evaluation.  In the opposite direction, when a
function call is about to return, the rule is
\[
\infer{%
  \langle\textsf{St}(\texttt{return}
  (\textsf{Ex}(\underline{(v,\tau)}, \mathit{se})), c), \sigma_0\rangle
  \rightarrow
  \langle \textsf{Ex}(c\underline{(v,\tau)}, \mathit{empty\_se}),
  \sigma\rangle}{
  \mbox{$\mathit{se}$ is empty of side effects}}
\]
where $\sigma$ is identical to $\sigma_0$ except that the various maps
for local variables, types and \texttt{struct}s have been reverted to
the state they held before the function was entered.

Note how the argument to the return statement is itself an extended
expression.  This means that the argument will be tagged with
\textsf{Ex} initially, but may later evolve to be a statement and
continuation (tagged with \textsf{St}) if the return-expression
includes a function call.  This means that the rule for evaluation of
the argument of return is
\[
\infer{\langle \textsf{St}(\texttt{return}(e_0),c), \sigma_0\rangle
  \rightarrow
  \langle \textsf{St}(\texttt{return}(e),c),\sigma\rangle}{
  \langle e_0, \sigma_0\rangle \rightarrow \langle e,\sigma\rangle}
\]
where $e_0$ and $e$ may be statements or expressions.


\section{Future}

The next deliverable requires a basic treatment of inheritance and
virtual, sub-class based polymorphism (dynamic dispatch).  This must
subsequently be extended into a treatment of multiple inheritance.
For this reason, I propose to adopt the modelling of inheritance by
Wasserrab \emph{et al.}~\cite{wasserrab-nst-OOPSLA06}.  I believe this
will provide a complete story.

The other big features of \cpp{} are exceptions and templates.  I
believe exceptions will be relatively straightforward.  The only way
in which they extend the existing ways in which \texttt{break} and
\texttt{continue} statements are trapped is by the way in which the
exception's type can determine whether or not it is trapped.  The rule
for trapping \texttt{break} is
\[
\infer{\langle \textsf{St}(\texttt{Trap}(\texttt{Brk},\texttt{break}),
  c), \sigma\rangle\rightarrow
\langle \textsf{St}(\texttt{EmptyStmt}, c), \sigma\rangle}{}
\]
If an exception of a particular type $\tau$ is to be trapped, and a
piece of code executed on the trap occuring, then we may be able to
model this with a rule of the form
\[
\infer{\langle\textsf{St}(\texttt{Trap}(\texttt{Exn}(\tau_1,\mathit{var},s),\texttt{exnval}\underline{(\mathit{val},\tau_2)}),
  c), \sigma_0\rangle \rightarrow
  \langle\textsf{St}(s,c),\sigma\rangle}{\tau_2 \leq \tau_1}
\]
where $\sigma$ is updated with the binding from variable
$\mathit{var}$ to exception value $\mathit{val}$, and where the $\leq$
relation in the hypothesis encodes the sub-class relation.  The
syntax for expressions will also need to be extended so that exception
values are also considered valid exceptions.


Templates will be rather more difficult.  In particular, it is still
not clear to me whether or not to try to reflect the fact that
considerable template ``activity'' is usually held to occur at
compile-time.  I suspect it will be better to model it as happening as
part of the usual dynamic semantics.  This may then give the
(misleading?) impression that what most implementations would treat as
a compile-time error might occur at run-time.

\subsection{Future Form}
I am open to suggestions as to how future deliverables should be
packaged and presented.

\bibliographystyle{plain}
\bibliography{deliverables}

\end{document}


%%% Local Variables:
%%% mode: latex
%%% TeX-master: t
%%% End:
