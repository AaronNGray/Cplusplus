\documentclass[11pt]{article}

\usepackage{charter}
\usepackage{alltt}
\usepackage{url}
\usepackage{proof}
\usepackage{amsfonts}
\usepackage{underscore}
\usepackage{pstricks}
\usepackage{pst-node}

\newcommand{\cpp}{\mbox{C\hspace{-0.5pt}+\hspace{-2pt}+}}
%\newcommand{\cpp}{C\nolinebreak\hspace{-.05em}\raisebox{.4ex}{\tiny\bf +}\nolinebreak\hspace{-.10em}\raisebox{.4ex}{\tiny\bf +}}

\newcommand{\ruleid}[1]{\textsl{[#1]}}

\newcommand\ckey[1]{\texttt{#1}}

\newcommand\statebrack[1]{\langle #1 \rangle}
\newcommand\exprarrow{\rightarrow_e}
\newcommand\tcexarrow{\exprarrow^*}
\newcommand\expreval[2]{\statebrack{#1} \exprarrow \statebrack{#2}}
\newcommand\tcexeval[2]{\statebrack{#1} \tcexarrow \statebrack{#2}}
\newcommand\stmtarrow{\rightarrow_s}
\newcommand\stmteval[2]{\statebrack{#1} \stmtarrow \statebrack{#2}}
\newcommand\vardarrow{\rightarrow_v}
\newcommand\vardeval[2]{\statebrack{#1}\vardarrow \statebrack{#2}}
\newcommand\hoaretriple[3]{\{#1\}\;#2\;\{#3\}}
\newcommand\hoarequad[4]{\{#1\}\;#2\;\{#3\}\{#4\}}

\newcommand\inferwhere[1]{#1\par\nopagebreak\vskip 1mm\nopagebreak%
                          \noindent\ignorespaces}

\newcommand\okval{\ckey{okVal}}
\newcommand\tyobj[1]{\operatorname{\textbf{obj}}[#1]}

\newcommand\addse{\ckey{add\_se}}
\newcommand\applyse{\ckey{apply\_se}}
\newcommand\clearse{\ckey{clear\_se}}
\newcommand\declvar{\ckey{decl\_var}}
\newcommand\initdeclvar{\ckey{init\_decl\_var}}
\newcommand\installparams{\ckey{inst\_parms}}
\newcommand\markref{\ckey{mark\_ref}}
\newcommand\memtrans{\ckey{mem\_trans}}
\newcommand\removerefs{\ckey{remove\_refs}}
\newcommand\structdeclare{\ckey{struct\_decl}}

\newcommand\cid{{\sf id}}
\newcommand\cgenop{\odot}
\newcommand\Cand{\ckey{\&\&}}
\newcommand\Cor{\ckey{||}}
\newcommand\ccond[3]{#1 \;\ckey{?}\; #2 \;\ckey{:}\; #3}
\newcommand\cassign[2]{#1\mathbin{\ckey{=}}#2}
\newcommand\ccast[2]{\ckey{(}#1\ckey{)}#2}
\newcommand\copassign[3]{#1\mathbin{#2\ckey{=}}#3}
\newcommand\andorsqpt{\widehat{\ckey{\&|}}}
\newcommand\cptr{\ptrtype}
\newcommand\se[2]{\clubsuit(#1,#2)}
\newcommand\rhsop{\diamondsuit}
\newcommand\ass[4]{%
  #2 \mathbin{\raisebox{-1.5mm}{$\stackrel{#4}{#1\ckey{=}}$}} #3}
\newcommand\cif[3]{%
\mbox{\ckey{if} \ckey{(}$#1$\ckey{)} $#2$ \ckey{else} $#3$}}
\newcommand\cloop{\mbox{$O$}}
\newcommand\trap{\mbox{$T$}}

\newcommand\declstmts{\mbox{\sl decls}\;\mbox{\sl stmts}}
\newcommand\cblock[1]{\mbox{\bf \{} #1\mbox{\bf \}}}
\newcommand\cwhile[2]{\mbox{\ckey{while} \ckey{(}$#1$\ckey{)} $#2$}}
\newcommand\cfor[4]{\mbox{\ckey{for ($#1$; $#2$; $#3$) $#4$}}}
\newcommand\cdowhile[2]{\mbox{\ckey{do $#1$ while ($#2$);}}}
\newcommand\tradwhile[2]{\mbox{\bf while}\;#1\;\mbox{\bf do}\;#2}
\newcommand\tradif[3]{\mbox{\bf if}\;#1\;\mbox{\bf then}\;#2\;%
                      \mbox{\bf else}\;#3}

\newcommand\ptrtype[1]{#1\ckey{*}}
\newcommand\arraytype[2]{#1\ckey{[}#2\ckey{]}}
\newcommand\sgndtype[1]{\ckey{signed}\;#1}
\newcommand\usgndtype[1]{\ckey{unsigned}\;#1}
\newcommand\void{\ckey{void}}
\newcommand\longdouble{\ckey{long~double}}
\newcommand\structtype[1]{\ckey{struct}\;#1}
\newcommand\struct{\ckey{struct}}

\newcommand\cnum{\ckey{Num}\,}
\newcommand\cfunref{\ckey{Funref}\,}
\newcommand\cnull{0}
\newcommand\cvar{\ckey{Var}\,}
\newcommand\caddr[1]{\ckey{\&}#1}
\newcommand\cderef{\ckey{*}\,}
\newcommand\cbreak{\ckey{break}}

\newcommand\cholundef{{\cal U}}
\newcommand\cvalue[2]{\underline{(#1,#2)}}
\newcommand\clvalue{\ckey{LV}}
\newcommand\cfvalue{\ckey{FV}}
\newcommand\crvreq{\ckey{RVR}}


\newcommand\retval{\ckey{RetVal}}
\newcommand\stmtval{\ckey{StmtVal}}
\newcommand\breakval{\ckey{BreakVal}}
\newcommand\contval{\ckey{ContVal}}

\newcommand\vardval{\ckey{VarDeclVal}}


\newcommand\wftype[2]{#1\vdash \text{\emph{wf}}(#2)}
\newcommand\etyping[3]{#1\vdash #2 : #3}

\newcommand\ctilde{\ckey{\~{}}}

\newcommand\lvtorv{\leadsto}
\newcommand{\funtype}[2]{#1 \rightarrow #2}
\newcommand{\SOME}[1]{\lfloor #1 \rfloor}

\newcommand{\last}{\textsf{last}}
\newcommand{\hd}{\textsf{hd}}

\newenvironment{proof}{\sl\setlength\parindent{0mm}\addtolength\parskip{2mm}}%
{}

\newenvironment{stdrule}{\begin{center}\begin{minipage}{\textwidth}\begin{alltt}}{\end{alltt}\end{minipage}\end{center}}

% Local Variables:
% mode: latex
% TeX-master: t
% End:


\newcommand{\ie}{\emph{i.e.}}
\newcommand{\naive}{na\"\i{}ve}

\title{Notes on Deliverable 4 (April 2007)}
\author{Michael Norrish\\{\small \texttt{Michael.Norrish@nicta.com.au}}}
\date{}


\begin{document}
\maketitle

\section{Form}

This deliverable consists of a compressed \texttt{tar}-file, that when
unpacked consists of a directory called \texttt{qinetiq-cpp}, which in
turn contains four directories
\begin{itemize}
\item \texttt{holsrcs}, containing the HOL source files of the
  mechanisation.  These files will build with the version of HOL4
  present in the CVS repository at SourceForge, with timestamp
  \texttt{2007-04-01 00:00Z}.  See Section~\ref{sec:getting-hol}
  below for instructions on how this version of HOL can be retrieved,
  and how the deliverable's HOL source files can then be built and
  checked.
\item \texttt{talks}, containing the \LaTeX{} source and a PDF for the
  talk presented at the DARP meeting in Newcastle in April 2006.  The
  source assumes that the \texttt{latex-beamer} and \texttt{PSTricks}
  packages are available.
\item \texttt{docs}, containing \LaTeX{} sources and a PDF version of
  this document, as well as sources for the notes on the earlier
  deliverables (nos.~1--3). 
\item \texttt{notes}, some \cpp{} source files that illustrate various
  aspects of \cpp{} behaviour.  An accompanying text file explains some
  of the behaviours.
\end{itemize}

\subsection{Building HOL Source-Files}
\label{sec:getting-hol}

\paragraph{Getting HOL From SourceForge}

To get a particular, dated, version of the HOL4 sources from the CVS
repository, one must first issue the command

{\small
\begin{verbatim}
   cvs -d:pserver:anonymous@hol.cvs.sourceforge.net:/cvsroot/hol login
\end{verbatim}
}

When prompted for a password, just press \texttt{Enter} to send a null
response.  The check-out of source code from SourceForge can now
proceed.  The source code fits into 60MB.  Issue the command

{\small
\begin{alltt}
   cvs \textit{repository-spec} co -D \textit{date} hol98
\end{alltt}
}

\noindent where \textit{\ttfamily repository-spec} is the string

{\small
\begin{alltt}
   -d:pserver:anonymous@hol.cvs.sourceforge.net:/cvsroot/hol
\end{alltt}
}

\noindent (also used in the \texttt{login} command), and where
\textit{\ttfamily date} is the desired date, best specified as an
ISO~8601 string enclosed in double-quotes.  For example,
\texttt{"2006-06-30 04:05Z"}.

Once a copy of the sources have been downloaded, further commands can
be used to update this copy to correspond to different dates.  As long
as such commands are issued from within the \texttt{hol98} directory,
the repository specification can be omitted.  The update command is

{\small
\begin{alltt}
   cvs update -d -D \textit{date}
\end{alltt}
}

\paragraph{Installing HOL} Once the sources have been downloaded, the
installation instructions from the page at
\url{http://hol.sourceforge.net} should be followed to build a copy of
HOL.  An installation of the Moscow~ML compiler (v2.01) will also be
required.

\paragraph{Building Deliverable Sources}
When HOL4 has been installed, the \texttt{Holmake} program (found in
the \texttt{hol98/bin} directory) can be run in the \texttt{holsrcs}
directory to create and check the logical theories.

\section{Content}

This deliverable adds a treatment of \cpp{} exceptions and templates
to the existing semantics. 

\subsection{Exceptions}
\label{sec:exceptions}

As suggested in Deliverable~1, exceptions can indeed be modelled in a
way similar to the existing treatment of \texttt{return},
\texttt{break} and \texttt{continue}.  One difference is that
exceptions propagate further: the \texttt{return} ``value'' only
propagates up as far as a function call (within an expression).  In
contrast, an exception will continue to propagate up through the
call-stack until it hits a suitable handler.

\newcommand{\ethrow}{\texttt{throw}\ensuremath{_e}}
This much allows a preliminary sketch of the behaviour.  The
\texttt{throw} form is actually an expression, but we set things up so
that there is a statement-level version of \texttt{throw} as well, and
it will be this that propagates through statement syntax.  The
expression syntax is written \ethrow{}, and the rule
\ruleid{expression-throw-some} describes the behaviour when the
\texttt{throw} has an argument:
\[
\infer{
  \statebrack{\textsf{Ex}(\ethrow(e)), \sigma}
  \rightarrow
  \statebrack{\textsf{St}(\texttt{throw}(e),c) \sigma}}{}
\]
The variable $c$ represents the continuation that would normally
convert the result of the statement into a value to be inserted into a
containing expression tree.  Because \texttt{throw} values can't ever
turn into values until they initialise a handler, this $c$ can be
anything at all. 

At the statement level, the \texttt{throw} form takes an extended
expression as an argument.  This evaluates its argument as one might
expect (rule \ruleid{throw-expr-evaluation})
\[
\infer{\statebrack{\texttt{throw}\;e_0,\sigma_0}\stmtarrow\statebrack{\texttt{throw}\;e,\sigma}}{\statebrack{e_0,\sigma_0}\rightarrow\statebrack{e,\sigma}}
\]
When a \texttt{throw}'s expression has been completely evaluated, we
have something that can then propagate upwards through the abstract
syntax of statements.  Because of the way the rules for loops and
\texttt{if}-statements work, we only have to worry about how
\texttt{throw}-statements traverse \texttt{Block} values. 

This propagation uses the rule \ruleid{block-interrupted}:
\[
\infer{
  \statebrack{
    \mathbf{\{} \;[\,] \;(s\!::\!\mathit{sts})\,\mathbf{\}},
    \sigma
  }
  \stmtarrow
  \statebrack{\mathbf{\{} \;[\,]\; [s]\mathbf{\}},\sigma}}{
  \textsf{final\_stmt}(s) & s \neq \textsf{EmptyStmt} & \mathit{sts}
  \neq [\,]}
\]
That this is a block rule is indicated by the
$\mathbf{\{\,\}}$-delimiters.  The first argument inside the block on
both sides of the arrow is an empty list of declarations.  The
predicate \textsf{final\_stmt} is true of \texttt{throw} and
\texttt{return} statements with fully evaluated arguments, as well as
of \texttt{break}, \texttt{continue} and the \textsf{EmptyStmt} form.
The latter doesn't cause an interruption, so is excluded by the second
hypothesis to the rule.  The final hypothesis ensures that there isn't
an infinite loop on this rule.

A \naive{} version of the rule for exitting a block
(\ruleid{block-exit}) is
\[
\infer{
  \statebrack{\mathbf{\{}\;[\,]\;[s]\;\mathbf{\}},\sigma}
  \stmtarrow
  \statebrack{s,\sigma}
}{
  \textsf{final\_stmt}(s)
}
\]
If a block has a final statement as its only statement, then that
statement can be propagated out to the top-level of the abstract
syntax. We will return to this rule when we come to worry about the
interaction of exceptions and object lifetimes.

Because exceptions arise from expressions, the statement level rules
need to acknowledge this possibility.  Thus, this rule for
\texttt{if} (\ruleid{if-exception}): 
\[
\infer{
  \statebrack{\textsf{St}(\texttt{if}\;G\;t\;e, c), \sigma}
  \rightarrow
  \statebrack{\textsf{mk\_exn}(G,c),\sigma}
}{
  \textsf{is\_exnval(G)}
}
\]
where $\textsf{is\_exnval}(G)$ is true if $G$ is of the form
$\textsf{St}(\texttt{throw}(e), c_0)$, and $e$ has been fully
evaluated.  The function \textsf{mk\_exn} takes an exception value and
replaces its continuation information with something appropriate for
the level of the containing statement.  In this example, $c_0$ will be
replaced by $c$.

\subsection{Templates}
\label{sec:templates}


\section{Future}

The next deliverable calls for a treatment of exceptions and
templates.  My notes for deliverable 1 cover how I believe exceptions
will be handled.  Templates remain unclear.  I would also appreciate
advice on which of the wrinkles and omissions above are seen as
significant and worthy of further attention.

\bibliographystyle{plain}
\bibliography{deliverables}

\end{document}

%%% Local Variables:
%%% mode: latex
%%% TeX-master: t
%%% End:
