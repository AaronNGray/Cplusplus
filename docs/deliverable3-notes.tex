\documentclass[11pt]{article}

\usepackage{charter}
\usepackage{alltt}
\usepackage{url}
\usepackage{proof}
\usepackage{amsfonts}
\usepackage{underscore}
\usepackage{pstricks}
\usepackage{pst-node}

\newcommand{\cpp}{\mbox{C\hspace{-0.5pt}+\hspace{-2pt}+}}
%\newcommand{\cpp}{C\nolinebreak\hspace{-.05em}\raisebox{.4ex}{\tiny\bf +}\nolinebreak\hspace{-.10em}\raisebox{.4ex}{\tiny\bf +}}

\newcommand{\ruleid}[1]{\textsl{[#1]}}

\newcommand\ckey[1]{\texttt{#1}}

\newcommand\statebrack[1]{\langle #1 \rangle}
\newcommand\exprarrow{\rightarrow_e}
\newcommand\tcexarrow{\exprarrow^*}
\newcommand\expreval[2]{\statebrack{#1} \exprarrow \statebrack{#2}}
\newcommand\tcexeval[2]{\statebrack{#1} \tcexarrow \statebrack{#2}}
\newcommand\stmtarrow{\rightarrow_s}
\newcommand\stmteval[2]{\statebrack{#1} \stmtarrow \statebrack{#2}}
\newcommand\vardarrow{\rightarrow_v}
\newcommand\vardeval[2]{\statebrack{#1}\vardarrow \statebrack{#2}}
\newcommand\hoaretriple[3]{\{#1\}\;#2\;\{#3\}}
\newcommand\hoarequad[4]{\{#1\}\;#2\;\{#3\}\{#4\}}

\newcommand\inferwhere[1]{#1\par\nopagebreak\vskip 1mm\nopagebreak%
                          \noindent\ignorespaces}

\newcommand\okval{\ckey{okVal}}
\newcommand\tyobj[1]{\operatorname{\textbf{obj}}[#1]}

\newcommand\addse{\ckey{add\_se}}
\newcommand\applyse{\ckey{apply\_se}}
\newcommand\clearse{\ckey{clear\_se}}
\newcommand\declvar{\ckey{decl\_var}}
\newcommand\initdeclvar{\ckey{init\_decl\_var}}
\newcommand\installparams{\ckey{inst\_parms}}
\newcommand\markref{\ckey{mark\_ref}}
\newcommand\memtrans{\ckey{mem\_trans}}
\newcommand\removerefs{\ckey{remove\_refs}}
\newcommand\structdeclare{\ckey{struct\_decl}}

\newcommand\cid{{\sf id}}
\newcommand\cgenop{\odot}
\newcommand\Cand{\ckey{\&\&}}
\newcommand\Cor{\ckey{||}}
\newcommand\ccond[3]{#1 \;\ckey{?}\; #2 \;\ckey{:}\; #3}
\newcommand\cassign[2]{#1\mathbin{\ckey{=}}#2}
\newcommand\ccast[2]{\ckey{(}#1\ckey{)}#2}
\newcommand\copassign[3]{#1\mathbin{#2\ckey{=}}#3}
\newcommand\andorsqpt{\widehat{\ckey{\&|}}}
\newcommand\cptr{\ptrtype}
\newcommand\se[2]{\clubsuit(#1,#2)}
\newcommand\rhsop{\diamondsuit}
\newcommand\ass[4]{%
  #2 \mathbin{\raisebox{-1.5mm}{$\stackrel{#4}{#1\ckey{=}}$}} #3}
\newcommand\cif[3]{%
\mbox{\ckey{if} \ckey{(}$#1$\ckey{)} $#2$ \ckey{else} $#3$}}
\newcommand\cloop{\mbox{$O$}}
\newcommand\trap{\mbox{$T$}}

\newcommand\declstmts{\mbox{\sl decls}\;\mbox{\sl stmts}}
\newcommand\cblock[1]{\mbox{\bf \{} #1\mbox{\bf \}}}
\newcommand\cwhile[2]{\mbox{\ckey{while} \ckey{(}$#1$\ckey{)} $#2$}}
\newcommand\cfor[4]{\mbox{\ckey{for ($#1$; $#2$; $#3$) $#4$}}}
\newcommand\cdowhile[2]{\mbox{\ckey{do $#1$ while ($#2$);}}}
\newcommand\tradwhile[2]{\mbox{\bf while}\;#1\;\mbox{\bf do}\;#2}
\newcommand\tradif[3]{\mbox{\bf if}\;#1\;\mbox{\bf then}\;#2\;%
                      \mbox{\bf else}\;#3}

\newcommand\ptrtype[1]{#1\ckey{*}}
\newcommand\arraytype[2]{#1\ckey{[}#2\ckey{]}}
\newcommand\sgndtype[1]{\ckey{signed}\;#1}
\newcommand\usgndtype[1]{\ckey{unsigned}\;#1}
\newcommand\void{\ckey{void}}
\newcommand\longdouble{\ckey{long~double}}
\newcommand\structtype[1]{\ckey{struct}\;#1}
\newcommand\struct{\ckey{struct}}

\newcommand\cnum{\ckey{Num}\,}
\newcommand\cfunref{\ckey{Funref}\,}
\newcommand\cnull{0}
\newcommand\cvar{\ckey{Var}\,}
\newcommand\caddr[1]{\ckey{\&}#1}
\newcommand\cderef{\ckey{*}\,}
\newcommand\cbreak{\ckey{break}}

\newcommand\cholundef{{\cal U}}
\newcommand\cvalue[2]{\underline{(#1,#2)}}
\newcommand\clvalue{\ckey{LV}}
\newcommand\cfvalue{\ckey{FV}}
\newcommand\crvreq{\ckey{RVR}}


\newcommand\retval{\ckey{RetVal}}
\newcommand\stmtval{\ckey{StmtVal}}
\newcommand\breakval{\ckey{BreakVal}}
\newcommand\contval{\ckey{ContVal}}

\newcommand\vardval{\ckey{VarDeclVal}}


\newcommand\wftype[2]{#1\vdash \text{\emph{wf}}(#2)}
\newcommand\etyping[3]{#1\vdash #2 : #3}

\newcommand\ctilde{\ckey{\~{}}}

\newcommand\lvtorv{\leadsto}
\newcommand{\funtype}[2]{#1 \rightarrow #2}
\newcommand{\SOME}[1]{\lfloor #1 \rfloor}

\newcommand{\last}{\textsf{last}}
\newcommand{\hd}{\textsf{hd}}

\newenvironment{proof}{\sl\setlength\parindent{0mm}\addtolength\parskip{2mm}}%
{}

\newenvironment{stdrule}{\begin{center}\begin{minipage}{\textwidth}\begin{alltt}}{\end{alltt}\end{minipage}\end{center}}

% Local Variables:
% mode: latex
% TeX-master: t
% End:


\newcommand{\ie}{\emph{i.e.}}

\title{Notes on Deliverable 3 (January 2007)}
\author{Michael Norrish\\{\small \texttt{Michael.Norrish@nicta.com.au}}}
\date{}


\begin{document}
\maketitle

\section{Form}

This deliverable consists of a compressed \texttt{tar}-file, that when
unpacked consists of a directory called \texttt{qinetiq-cpp}, which in
turn contains four directories
\begin{itemize}
\item \texttt{holsrcs}, containing the HOL source files of the
  mechanisation.  These files will build with the version of HOL4
  present in the CVS repository at SourceForge, with timestamp
  \texttt{2007-01-25 00:00Z}.  See Section~\ref{sec:getting-hol}
  below for instructions on how this version of HOL can be retrieved,
  and how the deliverable's HOL source files can then be built and
  checked.
\item \texttt{talks}, containing the \LaTeX{} source and a PDF for the
  talk presented at the DARP meeting in Newcastle in April 2006.  The
  source assumes that the \texttt{latex-beamer} and \texttt{PSTricks}
  packages are available.
\item \texttt{docs}, containing \LaTeX{} sources and a PDF version of
  this document, as well as sources for the notes on Deliverables~1
  and~2.
\item \texttt{notes}, some C++ source files that illustrate various
  aspects of C++ behaviour.  An accompanying text file explains some
  of the behaviours.
\end{itemize}

\subsection{Building HOL Source-Files}
\label{sec:getting-hol}

\paragraph{Getting HOL From SourceForge}

To get a particular, dated, version of the HOL4 sources from the CVS
repository, one must first issue the command

{\small
\begin{verbatim}
   cvs -d:pserver:anonymous@hol.cvs.sourceforge.net:/cvsroot/hol login
\end{verbatim}
}

When prompted for a password, just press \texttt{Enter} to send a null
response.  The check-out of source code from SourceForge can now
proceed.  The source code fits into 60MB.  Issue the command

{\small
\begin{alltt}
   cvs \textit{repository-spec} co -D \textit{date} hol98
\end{alltt}
}

\noindent where \textit{\ttfamily repository-spec} is the string

{\small
\begin{alltt}
   -d:pserver:anonymous@hol.cvs.sourceforge.net:/cvsroot/hol
\end{alltt}
}

\noindent (also used in the \texttt{login} command), and where
\textit{\ttfamily date} is the desired date, best specified as an
ISO~8601 string enclosed in double-quotes.  For example,
\texttt{"2006-06-30 04:05Z"}.

Once a copy of the sources have been downloaded, further commands can
be used to update this copy to correspond to different dates.  As long
as such commands are issued from within the \texttt{hol98} directory,
the repository specification can be omitted.  The update command is

{\small
\begin{alltt}
   cvs update -d -D \textit{date}
\end{alltt}
}

\paragraph{Installing HOL} Once the sources have been downloaded, the
installation instructions from the page at
\url{http://hol.sourceforge.net} should be followed to build a copy of
HOL.  An installation of the Moscow~ML compiler (v2.01) will also be
required.

\paragraph{Building Deliverable Sources}
When HOL4 has been installed, the \texttt{Holmake} program (found in
the \texttt{hol98/bin} directory) can be run in the \texttt{holsrcs}
directory to create and check the logical theories.

\section{Content}

This deliverable extends the existing HOL semantics for deliverable
2's ``\cpp-like'' language by adding treatments of \texttt{const},
multiple inheritance and constructor and destructor functions.  In
addition a number of other wrinkles are addressed (see
Section~\ref{sec:wrinkles} below).  The existence of the latter
suggests that a more detailed plan of work for other language features
should be drawn up.

\subsection{Postponing Statics}

There is a great deal of language in the standard encoding complicated
rules that are entirely static in their scope.  In this context,
``static'' means that these are rules that will be implemented by
compilers, and that programs that fail to be adhered to them will be
flagged as such at compile time.

For example, one such rule is the requirement that data members in
classes have unique names.  (Multiple functions can share the same
name because of overloading.)  Pertinently for this deliverable, there
are also a great meany rules to do with the ways in which
\texttt{const} pointers and reference can or can not be used, which
are all enforced by compilers.  As per the statement of work,
the deliverables will tend to ignore these constraints, and simply
assume that all programs are legal, and have been accepted by a
compiler.


\subsection{Multiple Inheritance}
\label{sec:multiple-inheritance}

As suggested in earlier deliverables, multiple inheritance has been
modelled by following the approach described in Wasserrab \emph{et
  al}~\cite{wasserrab-nst-OOPSLA06}.  The changes supporting this
addition to the model are relatively minor and almost all appear in
the theory \texttt{class_info}, where a number of new definitions
appear.

\newcommand{\Cs}{\mathit{Cs}} \newcommand{\fld}{\mathit{fld}}

At the top level, the rule for calculating the function that will be
called dynamically (\ruleid{function-member-select}) has changed
slightly.  It now reads
\[
\infer{
  \begin{array}{l}
    \statebrack{\textsf{Ex}((\clvalue(a,\texttt{Class}\;C,\Cs))\;\mathbf{.}\;
      \fld,\mathit{se}),\sigma} \rightarrow \\
    \qquad\left\langle\textsf{Ex}\left(\cfvalue\left(
        \begin{array}{l}
          \texttt{MFn}\;(\last(\Cs'))\;\fld,\\
          (\funtype{\mathit{argtys}}{\mathit{dynret}}),\\
          \SOME{(\clvalue\;a\;(\texttt{Class}\;C)\;\Cs')}
        \end{array}\right),
      \mathit{se}\right),\sigma\right\rangle
  \end{array}}{
  \begin{array}{c}
    \sigma \vdash \last(\Cs) \textrm{ has least method } \fld :
    \funtype{\mathit{statargs}}{\mathit{statret}} \textrm{ via } Ds \\
    \sigma \vdash (C,\Cs \,@_p\, Ds) \textrm{ selects } \fld :
    \funtype{\mathit{dynargs}}{\mathit{dynret}} \textrm{ via } \Cs'
  \end{array}}
\]
This rule describes how the call to method $\fld$ is resolved
for an object located at address $a$, with dynamic type $C$, and where
the static type of the object is the last element of the list $\Cs$.
The list $\Cs$ is also a path through the class hierarchy, starting at
the dynamic type and ending at the current static type.  (Note that an
object's dynamic type is determined on object creation, and persists
for an object's entire life-time.  In contrast, an object's static
type is the type ascribed to it by a particular piece of code.
Different pieces of code may well ``see'' the same object as having
different types.  In this sense, an object's dynamic type is
unchanging, but it will have a variety of static types across the text
of a program.  Confusing, but true!)

The first premise (``has-least-method'') examines the static type of
the object for which the method will be called, $\last(\Cs)$.
Starting at that point in the hierarchy it looks upwards (\ie, towards
base classes) for the nearest base class that provides an
implementation of the desired method.  This base-class might be
$\last(\Cs)$ itself, in which case the path found ($Ds$) will be
empty.  There must also be a unique closest ancestor providing the
desired method.  If this isn't the case, then there will have been a
compile-time error, as the call will be statically ambiguous.

The second premise (``selects-via'') then determines the dynamic
location of the desired method.  There are two cases.  The simple case
is when there is a unique best method for the dynamic type, $C$.
Imagine, for example, that there is a four-element singly-linked
inheritance graph, from base $B_0$ down to most-derived $B_3$, with
implementations of the method $\fld$ at $B_0$ and $B_2$.  If the
object is actually of type $B_3$, but is statically seen as type
$B_1$, then $C$ is $B_3$, and $\Cs$ will be $[B_3,\,B_2,\,B_1]$.  The
first premise (``has-least-method'') determines that, starting at
$B_1$ (the static type) there is an implementation of $\fld$ at path
$[B_1, B_0]$.  This will be the instantiation of the rule's variable
$Ds$.

The simple case for ``selects-via'' checks whether or not there is
also a least method for the dynamic type (ignoring, for the moment,
the path giving the static type).  Thus
\[
\infer{\sigma \vdash (C,\Cs) \textrm{ selects } \fld \textrm{ via } \Cs'}{\sigma \vdash C \textrm{ has least method } \fld :
  \funtype{\mathit{argtys}}{\mathit{retty}} \ \textrm{ via } \Cs'}
\]
In the simple example, we will thus conclude that \[
\sigma \vdash (B_3,[B_3,B_2,B_1,B_0]) \textrm{ selects } \fld
\textrm{ via } [B_3,B_2]
\]
so that the call will be to the implementation of $\fld$ in $B_2$, and
the \texttt{this} pointer will be adjusted so that the type and path
information associated with its value will be $(B_3,[B_3,B_2])$ (a
dynamic type of $B_3$, as always, and a static type of $B_2$).

The same rule applies in a much more complicatd seeming situation,
where multiple inheritance and shared base objects come into play.
Consider the program in Figure~\ref{fig:diamond-cpp}.  When the call
to \texttt{cref.f()} is made, the type and path associated with the
reference will be $(D,[D,C_1])$.  Statically, the reference is a
\texttt{C1} value, but dynamically, it's really of class
\texttt{D}. The first premise in rule \ruleid{function-member-select}
finds that there is a path ($\mathit{Ds}$ in the rule) which is
appropriate for \texttt{f}.  This path is $[B]$.  The fact that the
path does not include the derived object's name, and is just a bare
reference to a class indicates that it is a path to a shared base.
When such a path is the second argument of path concatenation (the
$@_p$ operator in the second premise), the result is just the second
argument, so that the second hypothesis in the rule becomes
\[
\sigma \vdash (D,[B]) \textrm{ selects } f :
\funtype{\texttt{void}}{\texttt{int}} \textrm{ via } \Cs'
\]
\begin{figure}[hbtp]
\begin{verbatim}
#include <iostream>

class B {
public:
  virtual int f() { std::cout << "B's f\n"; return 3; }
  virtual ~B() { }
};

class C1 : virtual public B {
};

class C2 : virtual public B {
public:
  virtual int f() { std::cout << "C2's f\n"; return 4; }
};

class D : public C1, public C2 {
};

int dosomething(C1 &cref)
{
  return cref.f();
}

int main()
{
  D d;
  return dosomething(d);
}
\end{verbatim}
\label{fig:diamond-cpp}
\caption{Multiple inheritance with shared base objects.  (This program is
in the \texttt{notes} directory with name \texttt{diamond-multinherit.cpp}.)}
\end{figure}

The first, simple, rule for ``selects-via'' resolves this.  Ignoring
the path $[B]$, the simple rule checks whether or not there is a
unique least path to an \texttt{f} from $D$.  There is such a path,
and it is $[D,C_2]$.  So, the call to \texttt{cref.f()} ends up being
a call to the \texttt{f} in class $C_2$, in an ``unrelated'' part of
the object hierarchy.
\[
\rule{0.2\textwidth}{.1mm}
\]

For those situations where there is not a unique path from the dynamic
type to a best selection, there is another more complicated rule
defining ``selects-via''.  In Figure~\ref{fig:lopsided-v}, the
inheritance hierarchy looks like
\[
\psset{rowsep=3ex,nodesep=2mm,colsep=2em}
\begin{psmatrix}
\texttt{\psframebox{B}} \\
\texttt{Left1}\\
\texttt{\psframebox{Left2}} & & \texttt{\psframebox{Right}}\\
& \texttt{D}
\end{psmatrix}
\ncline{1,1}{2,1}
\ncline{2,1}{3,1}
\ncline{3,1}{4,2}
\ncline{3,3}{4,2}
\]
where the boxed class names are those that implement the function
\texttt{f}.  The question is which \texttt{f} will be called when the
dynamic type is \texttt{D}, and when the static type is
\texttt{Left1}.  There is no unique, least implementation of
\texttt{f} visible from the dynamic type (\texttt{D}), so the simple
rule does not apply.  Instead, the notion of \emph{overrider} is
introduced via the rule
\[
\infer{\sigma\vdash (C,\Cs) \textrm{ selects }\fld : m \textrm{ via }
  \Cs'}{
\begin{array}{c}
  \forall m\;\Cs'. \;\;\neg \;\sigma \vdash C \textrm{ has least }\fld : m
\textrm{ via } \Cs'\\
\sigma \vdash (C,\Cs) \textrm{ has overrider } \fld : m \textrm{ via }
\Cs'
\end{array}}
\]
where the static type of the value again plays a role.  In this case,
the \texttt{f} that gets called is that in \texttt{Left2}.  For
further details on exactly how this is defined, see either
\texttt{class_infoScript}, or~\cite{wasserrab-nst-OOPSLA06}.

\begin{figure}[hbtp]
\begin{verbatim}
#include <iostream>

class B {
public:
  virtual void f() { std::cout << "B's f\n"; }
  virtual ~B() { }
};

class Left1 : public B { };

class Left2 : public Left1 {
public:
  virtual void f() { std::cout << "Left2's f\n";  }
};

class Right {
public:
  virtual void f() { std::cout << "Right's f\n"; }
  virtual ~Right() { }
};

class D : public Left2, public Right { };

void dosomething(Left1 &l1ref) { l1ref.f(); }

int main()
{
  D d;
  // d.f();  would be statically ambiguous
  dosomething(d);
  return 0;
}
\end{verbatim}
\caption{A lop-sided `V' inheritance hierarchy.  Available as
  \texttt{notes/lopsided-v.cpp}.}
\label{fig:lopsided-v}
\end{figure}

\subsection{Object Lifetimes}
\label{sec:object-lifetimes}

\paragraph{Constructors}
Handling constructors has easily wrought the greatest change to the
previous deliverable's model.  The basic approach taken to objects has
been to encode as much as possible with ``evolving syntax''.  Just as
\texttt{while} can be modelled by having
\[
\texttt{while}\;\texttt{(}G\texttt{)}\;\mathit{body}
\] become \[
\texttt{if}\;\texttt{(}G\texttt{)}\;\texttt{\{}\mathit{body}\texttt{;}\;\texttt{while}\;\texttt{(}G\texttt{)}\;\mathit{body} \texttt{\}}
\]
so too do calls to \cpp{} constructors create new programs ``in
place''.  The advantage of this approach is that there is less need
for relatively complicated state to be recorded in yet more fields in
the big record type \texttt{state} (in \texttt{statesScript}).
Instead, programs can unfold into more elaborate forms directly.  The
disadvantage of this approach is that the original syntax may not
support enough forms, requiring new constructors to be created, or
extended with new parameters.

The existing handling of block-statements is an example of this latter
disadvantage.  In particular, the abstract syntax has a constructor
\texttt{Block} with type
\[
\texttt{:bool -> decl list -> stmt list}
\]
where the boolean flag indicates whether or not the block has been
entered yet.  In this way, the abstract syntax has values in it that
can't be written down in the concrete syntax.  (Simiarly, the original
\textsf{Cholera} model for C included the \texttt{RVR} constructor and
the $\hat{f}$ intermediate form for function calls.)

Constructors are intimately tied up with declarations and
initialization.  In the simple C world, a variable comes into being in
two stages.  First space in memory is allocated for the variable
(whether on the stack or heap), and the variable name is associated
with that space for the span of the variable's life.  Then there is an
optional initialization phase, when the piece of memory associated
with the variable space is filled in with some value.

The \cpp{} model is similar.  All new objects (but not references)
must be associated with some new space.  Then they may or may not be
initialised.  Additionally, in \cpp{}, an object that is only
declared, and which does not appear to be initialised, will actually
have its default constructor called.

The abstract syntax supporting this is all defined in the HOL source
file \texttt{statementsScript}:

\begin{center}
\begin{tabular}{lll}
Constructor & Argument Types & Description\\
\hline
\texttt{VDec} & $\mathit{name},\mathit{type}$ & no initialization\\
\texttt{VDecInit} &
$\mathit{name},\mathit{type},\mathit{initializer}$ &
initialization (unallocated) \\
\texttt{VDecInitA} &
$\mathit{varlocation},\mathit{type},\mathit{initializer}$ &
initialization (space allocated)
\end{tabular}
\end{center}

For example, when a class is declared with no explicit initialization
of any sort, meaning that the default constructor will be called, the
syntax moves through all three stages.  The abstract syntax
corresponding to something like
\begin{verbatim}
{
   classname c;
   ...
}
\end{verbatim}
will be \texttt{VDec "c" classname}.  Because this is a class type,
rule \ruleid{decl-vdec-class} will fire (and can do so immediately),
and the declaration will become \texttt{VDecInit "c" classname init},
where the form of \texttt{init} will encode the fact that a direct
initialization is being performed, and that there are no arguments.
Then the rule \ruleid{decl-vdecinit-start-evaluate-direct-class}
fires.  Side conditions of this rule cause space to be allocated (at
address \texttt{a}), the state's maps from names to addresses be
updated, and for the construction of the class to be recorded so that
it can be destroyed later.

The syntax also evolves to become
\begin{verbatim}
   VDecInitA classname (ObjPlace a) init'
\end{verbatim}
where the new initializer records that a function call to a
constructor is about to happen, and where that construction will
happen (\ie, \texttt{init'} includes a reference to \texttt{a}).

There are three forms of initializer.  The first two are
\texttt{DirectInit0} and \texttt{DirectInit} and correspond to direct
initialization~\cite[\S 8.5 paragraph 12]{cpp-standard-iso14882}.  The
\texttt{DirectInit0} constructor takes as arguments a list of
expressions.  In this way, concrete syntax such as
\begin{verbatim}
{
  Classname c(x,&y,z+1);
}
\end{verbatim}
is directly modelled.  (When the rule \ruleid{decl-vdec-class} fires,
the newly created \texttt{DirectInit0} constructor takes an empty list
of arguments.)

The \texttt{DirectInit} constructor takes one argument, an ``extended
expression'', which will initially be an expression constructed by an
application of the special constructor \texttt{ConstructorFVal} to the
same argument list.  This form needs to be an extended expression so
that the body of the constructor (a statement) can be entered.

The other form of initializer is constructed by the function
\texttt{CopyInit}.  This corresponds to the syntax
\begin{verbatim}
{
  type varname = expression;
  ...
}
\end{verbatim}
which is a
copy-initialization~\cite[\emph{ibid}]{cpp-standard-iso14882}.  When
the type of the new object is not a class, there is no difference
between copy-initialization and direct-initialization, reflected in
the rule \ruleid{decl-vdecinit-start-evaluate-direct-nonclass}, which
moves from a \texttt{DirectInit0} to a \texttt{CopyInit} initializer.

When a \texttt{CopyInit} initializer completes its evaluation,
yielding a value, that value be copied across into the space earlier
allocated for the object.  For non-class types this is done with the
same \texttt{val2mem} helper function that is used to apply side
effects.  For class types, this copying must be performed by a call to
the copy constructor.

\paragraph{Constructor Calls}
The expression form corresponding to a constructor call uses the
expression form for function applications, but with a special form in
the place of the function value.  This allows the normal evaluation of
function applications (with the unspecified order of evaluation of
arguments, for example).  The function value position is filled by a
new abstract syntactic form \texttt{ConstructorFVal}.  This takes
three parameters:
\begin{itemize}
\item a boolean indicating whether or not the constructor is being
  called for a most-derived object or not;
\item the address of the space which the constructor is to operate
  over; and
\item the name of the class that is being constructed
\end{itemize}
So, if a declaration is made of the form
\begin{verbatim}
{
   class v(x);
}
\end{verbatim}
then, once sufficient space is allocated for the new object \texttt{v}
(at address \texttt{a}, say), the abstract syntax will look like
\begin{verbatim}
  VDecInitA "class"
            (ObjPlace a)
            (DirectInit
               (NormE (FnApp (ConstructorFVal T a "class")
                             [Var "x"])
                      base_se))
\end{verbatim}
The \texttt{ObjPlace} constructor is used to distinguish this from the
situation where a reference is being initialised.  The \texttt{NormE}
constructor specifies that the current form is an expression (as
opposed to the statement that will be in this position once the
constructor body is entered).  The \texttt{base_se} value is the empty
side-effect record.  This will evolve as references to and updates of
memory occur.

Once the parameters to the constructor have been evaluated, the
constructor body can be entered.  This happens in rule
\ruleid{constructor-function-call}:
\[
\infer{
  \begin{array}{c}
    \statebrack{\textsf{Ex}(\texttt{FnApp_sqpt}\;
      (\texttt{ConstructorFVal}\;\mathit{mdp}\;\mathit{addr}\;C)\; \vec{a},
      \mathit{se}_0),\sigma_0)}\\
    \longrightarrow\\
    \left\langle
      \begin{array}{l}
        \textsf{St}\left(
          \begin{array}{l}
            \texttt{Block}\;\bot\;(\mathit{pdecls}\frown\mathit{cpfx})\;b,\\
            \texttt{return_cont}\;\mathit{se}_0\;\texttt{Void}
          \end{array}\right),\\
        \sigma_0 \textrm{ with }
        \texttt{thisvalue := }\lfloor\underline{(\mathit{this},(C\texttt{*}))}\rfloor
      \end{array}
    \right\rangle
  \end{array}}{
  \begin{array}{l}
    \texttt{find_constructor_info}\;\sigma_0\;C\;\vec{a}\;\vec{p}\;\vec{\mathit{m\imath}}\;b\\
    \!\!\begin{array}{l}
      \mathit{pdecls} =\\
      \quad\textsf{map}
      \begin{array}[t]{l}
        (\lambda((n,\tau),a).\;
        \texttt{VDecInit}\;\tau\;n\;(\texttt{CopyInit}(\textsf{Ex}(a,\texttt{base_se})))) \\
        (\textsf{zip}(\vec{p},\vec{a}))
      \end{array}
    \end{array}\\
    \lfloor \mathit{this}\rfloor =
    \texttt{ptr_encode}\;\sigma_0\;\mathit{addr}\;C\;[C]\\
    \mathit{cpfx} = \texttt{construct_ctor_pfx}\;\sigma_0\;\mathit{mdp}\;\mathit{addr}\;C\;\vec{\mathit{m\imath}}
  \end{array}}
\]
This is complicated enough, and there is more complexity hidden behind
the auxiliary functions and relations.  The first auxiliary is the
relation \texttt{find_constructor_info}, which appears in the rule's
first hypothesis.  This relation treats its first three parameters
($\sigma_0$, $C$ and $\vec{a}$) as inputs.  These are the current
state, the name of the class being constructed, and the actual
arguments being passed to the constructor.  The remaining arguments to
the relation are ``outputs''.  The vector $\vec{p}$ is the list of
formal parameters (names and types).  The vector
$\vec{\mathit{m\imath}}$ is the list of ``mem-initializers''
(see~\cite[\S12.6.2]{cpp-standard-iso14882}) associated with the
constructor, and $b$ is the constructor's body.  The
\texttt{find_constructor_info} auxiliary is responsible for resolving
which constructor needs to be called, based on the types of the actual
arguments.

The second hypothesis of the rule constructs the sequence of variable
declarations corresponding to the parameters, using standard
functional programming auxiliaries \textsf{map} and \textsf{zip}.
Parameter passing is just like variable declaration\footnote{This does
  away with the \textsf{Cholera} approach which had a number of
  auxiliary relations effectively duplicating what occurred in
  variable declaration}, and so the model's existing treatment of
declarations can be re-used to set up the binding between formal names
and actual values.  Note that the expressions that initialise the
parameters have already been fully evaluated, so that there will be no
expression evaluation done when the declarations come to be evaluated
(except for any class construction that may be called for).

The third hypothesis calculates a value for the \texttt{this} value.
The dynamic and static type of the \texttt{this} pointer will be the
same (as the pair $C$ and $[C]$ are passed to \texttt{ptr_encode}),
and there will not be any polymorphic dispatch to functions in derived
classes if any virtual functions are called in the constructor bodies.

The fourth hypothesis constructs a ``c-prefix'' of declaration calls
to initialise class members and bases.  This is all done in the
complicated function \texttt{construct_ctor_pfx} (defined in
\texttt{dynamicsScript}).  This constructs a sequence of declarations
to initialise the non-static members of the new class, and the class's
immediate bases.  The mem-initializers are consulted to see what
initializers should be provided.  (If a mem-initializer is not
provided for a given member or base, then that object will be value-
or default-initialized; see~\cite[\S12.6.2, paragraphs
3--4]{cpp-standard-iso14882}.)

For example, in Figure~\ref{fig:mem-inits}, before class \texttt{C}'s
constructor body is even entered, the parameters \texttt{cptr} and
\texttt{i} need to be declared and initialised with actual values.
Subsequently, all of \texttt{C}'s immediate bases (just \texttt{B} in
this case) need to be constructed, followed by its members
(\texttt{ptr} and \texttt{sz}).  Note that while the paramters need to
have space allocated for them, the bases and members do not (because
the space for the entire object was allocated at the \texttt{VDecInit}
stage).

\begin{figure}[htbp]
\begin{verbatim}
#include <cstring>

class B {
  int x;
public:
  B(int i) : x(i) {}
};

class C : public B {
  char *ptr;
  int sz;
public:
  C(char *cptr, int i)
    : B(cptr[i]), ptr(cptr), sz(strlen(cptr)) { }
};
\end{verbatim}
\caption{\cpp{} constructors with \emph{mem-initializers}.  (Available
as \texttt{notes/mem-inits.cpp}.)}
\label{fig:mem-inits}
\end{figure}

Assume that the constructor has been called with parameters \texttt{x}
and \texttt{y}, and that these have evaluated to values \texttt{xval}
and \texttt{yval}. The sequence of declarations that are constructed
to precede the constructor body is given in
Figure~\ref{fig:constructor-vdecs}. The first two declarations are of
the parameters.  The next constructs the base \texttt{B}, and the last
two construct the non-static members.  The body of
\texttt{construct_ctor_pfx} is responsible for calculating the offsets
of the members (given as \texttt{Boff}, \texttt{ptroff} and
\texttt{szoff} in the figure.

\begin{figure}[hbtp]
\begin{verbatim}
  VDecInit (char *) cptr (CopyInit (NormE xval base_se))
  VDecInit int      i    (CopyInit (NormE yval base_se))

  VDecInitA B (ObjPlace (a + Boff))
              (DirectInit
                 (NormE
                    (FnApp (ConstructorFVal F (a + Boff) B)
                           [Deref(Plus (Var "cptr")
                                       (Var "i"))])
                    base_se))

  VDecInitA (char *) (ObjPlace (a + ptroff))
                     (CopyInit (NormE (Var "cptr") base_se))
  VDecInitA int      (ObjPlace (a + szoff))
                     (CopyInit (NormE (FnApp (Var "strlen")
                                             [Var "cptr"])
                                      base_se))
\end{verbatim}
\caption{The variable declarations constructed to precede the body of
  C's constructor (from Figure~\ref{fig:mem-inits}).}
\label{fig:constructor-vdecs}
\end{figure}

Note how the first argument of the \texttt{ConstructorFVal} form in
the construction of the base \texttt{B} is false; this is because
\texttt{B} is not the most-derived object.  If there were any shared
bases in the example, the most-derived object would be ``responsible''
for constructing them (see~\cite[\S12.6.2, paragraph
5]{cpp-standard-iso14882}).  If \texttt{C} had any non-static members
of class type, then these would be constructed with their
$\mathit{mdp}$ flag set to true.

When the constructor for the base class \texttt{B} comes to be called,
it will in turn initialise its members.  The constructor for
\texttt{B} is called with an argument (\texttt{cptr[i]}) that needs to
be evaluated in the context where the parameters are in scope, so it
is clear that the declarations for the parameters must come before the
base and member initialisations.

\paragraph{Object Destruction}
When an object of class type is first declared (with a
\texttt{VDecInit} form), it has memory allocated sufficient to contain
the new class in its entirety (including sub-objects).  This
allocation is reflected in the state's \texttt{allocmap}.  When the
block in which this declaration was made is left, this allocation is
forgotten.  At the same time, the destructor for the object must be
called.  This is modelled with a new field for the state,
\texttt{blockclasses}.  This is of type
\begin{verbatim}
   (addr # CPPname # CPPname list) list list
\end{verbatim}
The address is the address of the object which is to be destroyed, and
the two remaining are the class name and path which serve to identify
the object's type.  The inner list constructor reflects the fact that
multiple objects may be declared at the same scope level.  The outer
list constructor appears because of the stack discipline necessary for
multiple nested scopes.

When an object of class type is declared, the rule responsible
(\ruleid{decl-vdecinit-start-evaluate-direct-class}) both allocates
the necessary memory, and adds address and type information for all of
the object's sub-classes and members (not just immediate bases, but
doing a complete traversal of the inheritance graph) to the
\texttt{blockclasses} field.  This is done by the
\texttt{update_blockclasses} relation defined in
\texttt{class_infoScript}.  The object information has to be added
to the list in the correct order so that objects will be destroyed in
reverse order of construction.

The rule for exitting from a block is then adjusted so that it can
only occur if the current scope's \texttt{blockclasses} information is
empty.  As long as it is not empty, the destructor corresponding to
the object on the top of the stack is set up to be called.  This is
done in rule \ruleid{block-exit-destructor-call}:
\[
\infer{
  \begin{array}{c}
    \statebrack{
      \textsf{St}(\texttt{Block}\;\top\;[]\;[\mathit{st}],c),\sigma}\\
    \longrightarrow\\
    % resulting state
    \left\langle
      \begin{array}{l}
        \textsf{St}\left(\texttt{Block}\;\top\;[]\;
          \left[
            \begin{array}{l}
              \texttt{Alone}
              (\textsf{Ex}(\texttt{DestCall}\;a\;\mathit{cnm}),\texttt{base_se}),\\
              \mathit{st}
            \end{array}\right],
          c\right),\\
        \sigma\textrm{ with }\texttt{blockclasses := }
        \mathit{rest} :: \mathit{bcs}\\
      \end{array}
    \right\rangle % end of resulting state
  \end{array}
  }{\texttt{final_stmt}\;\mathit{st} &
    \sigma.\texttt{blockclasses} = ((a,\!\mathit{cnm},p) ::
    \mathit{rest}) :: \mathit{bcs}}
\]
This is another example of evolving syntax: the block that the flow of
control is about to leave, has this departure deferred with the
insertion of a new statement before the block's final statement.  The
rule allowing an exit to eventually occur is \ruleid{block-exit}:
\[
\infer{
  \statebrack{
    \textsf{St}(\texttt{Block}\;\top\;[]\;[\mathit{st}], c), \sigma_0}
  \longrightarrow
  \statebrack{\textsf{St}(\mathit{st}, c), \sigma}
}{
  \begin{array}{l}
    \sigma_0.\texttt{stack} =
    (\mathit{stm},\mathit{tym},\mathit{vrm},\mathit{this}) ::
    \mathit{stk}
    \\
    \texttt{final_stmt}\;\mathit{st}
    \\
    \sigma_0.\texttt{blockclasses} = [] :: \mathit{bcs}
    \\
    \sigma = \sigma_0 \textrm{ with } \left\langle\begin{array}{ll}
        \texttt{stack := }\mathit{stk}; &
        \texttt{classmap := }\mathit{stm};\\
        \texttt{typemap := }\mathit{tym}; &
        \texttt{varmap := }\mathit{vrm};\\
        \texttt{thisvalue := }\mathit{this}; &
        \texttt{blockclasses := }\mathit{bcs}
      \end{array}\right\rangle
  \end{array}
}
\]
The predicate \texttt{final_stmt} is true of a statement if it is
an abnormal exit form (such as \texttt{break} or \texttt{return}), or
simply \texttt{EmptyStmt}.

Actually calling a destructor is straightforward because there are no
parameters, nor anything comparable to the mem-initializers.  The
requirement in the standard that sub-objects be destroyed before the
body of a destructor is entered is handled by the fact that the
sub-object information is entered into the \texttt{blockclasses} stack
ahead of the final encompassing object.



\subsection{The Type-modifier \texttt{const}}
\label{sec:const-type-modifier}

Handling the \texttt{const} modifier is quite straightforward if the
model can ignore static issues.  In particular, there is really only
one place where \texttt{const} and the dynamic semantics interact: if
a program attempts to update memory that has been declared
\texttt{const} then this is undefined behaviour.

Types must admit the \texttt{const} modifier, so the algebra for the
abstract syntax of types is adjusted to include the new possibility:
\[
\begin{array}{lcl}
\tau & ::= & \texttt{Char} \;\;| \;\; \texttt{Ptr}\;\tau\;\;|\;\;\texttt{Array}\;\tau\;n\\
& | & \dots\\
& | & \texttt{Const}\; \tau
\end{array}
\]
Note the confusion possible because of the differences between this
syntax and \cpp's.  For example, what is written in \cpp{} as
``\texttt{const char *}'' is really ``\texttt{Ptr (Const Char)}''.  An
object of \texttt{const} type is one whose type has \texttt{Const} as
its outermost constructor or tag.  Thus \texttt{(const char *)} is
\emph{not} a \texttt{const} type; it's a pointer to one.

When an object of \texttt{const} type is declared, this marks some
part of memory as \texttt{const}.  This is handled by having a new
field in the model's state, a set of memory addresses called
\texttt{constmap}.  If an address is in this map, then it must not be
updated.  The HOL relation \texttt{vdeclare} (in
\texttt{declaration_dynamicsScript}) handles this situation.  This
relation also allocates memory, updating the \texttt{allocmap}.
Previously, the relevant part of \texttt{vdeclare}'s definition had
\[
  \sigma = \sigma_0 \textrm{ with } \langle\texttt{allocmap := }
  \sigma_0.\texttt{allocmap} \cup \mathit{rs}; \dots \rangle
\]
(where $\mathit{rs}$ is the set of addresses ``occupied'' by the new
object).

Now, the definition has
\[
\sigma = \begin{array}[t]{l}\sigma_0 \textrm{ with }\\
\;\;\left\langle
  \begin{array}{l}
    \texttt{allocmap := } \sigma_0.\texttt{allocmap} \cup \mathit{rs};\\
    \texttt{constmap := } \begin{array}[t]{l}
      \textsf{if} \;\texttt{const_type}(\tau)\;\textsf{then}\;
      \sigma_0.\texttt{constmap} \cup \mathit{rs}\\
      \textsf{else}\;\sigma_0.\texttt{constmap} - \mathit{rs};
    \end{array}\\
    \dots
  \end{array}\right\rangle
\end{array}
\]

This much ensures that parts of memory can become marked as
\texttt{const}.  The other dynamic change required is to cause
undefined behaviour if a \texttt{const} part of memory is updated.
This is also a very small change: the definition of \texttt{apply_se}
(in \texttt{side_effectsScript}) checks not only that a side effect is
not updating already updated memory, but now also checks that the side
effect is not updating memory marked as \texttt{const}.

The existing rules in the dynamic semantics are already set up to flag
undefined behaviour if it ever comes about that a side effect record
is not empty and that no side effect can be legally applied.

Note that the \texttt{constmap} does not need to be managed on a
stacked scope basis like the variable and type maps.  This is because
it is subsumed by the \texttt{allocmap}, and is adjusted as the
allocmap changes.  Any attempt to update part of memory outside the
\texttt{allocmap} will fail automatically, so its \texttt{cosnt}-ness
is irrelevant.

\subsection{Wrinkles and Omissions}
\label{sec:wrinkles}

\begin{description}
\item[Arithmetic Operators]  The shift operators are not specified,
  nor is arithmetic on floating point numbers.
\item[Cast Syntax] There is no syntax in the type of expressions to
  allow new-style \cpp{} casts (such as \texttt{dynamic_cast} and
  \texttt{static_cast}).
\item[Declarations in blocks] For the moment, the model still enforces
  the old C requirement that blocks begin with a sequence of
  declarations followed by a sequence of statements.  Declarations
  inside \texttt{if}, \texttt{for} and \texttt{while} guards are also
  not allowed.  The more liberal \cpp{} rules can be emulated by
  introducing fresh enclosing blocks everytime a declaration appears
  that would violate the old C rules.  (This is a dynamic emulation
  only; this approach would allow multiple declarations of the same
  variable at the same ``block level'' statically if nothing else
  prevented it.)
\item[Default arguments]  Completely omitted.
\item[The heap] This is handled only in very rudimentary fashion.
  There is support for simple \texttt{new} expressions (see, for
  example, rule \ruleid{new-nonclass}), but handling \texttt{delete}
  properly will require the state to record the types of what was
  allocated.  (When a call to \texttt{delete []} is made, the state
  needs to know how big the array was.) Allocation of heap memory is
  handled by having a \texttt{hallocmap} that records allocations that
  have been made on the heap.  Checks that in \textsf{Cholera} looked
  only at \texttt{allocmap} now look at the union of it and
  \texttt{hallocmap}.  Handling placement-\texttt{new} is likely to be
  rather intricate to specify.
\item[Nested and local classes] Omitted.  Neither poses a big
  challenge dynamically; they are much more interesting statically.
\item[Non-virtual member functions] The model assumes that all member
  functions are virtual.  Non-virtual member functions should be
  handled like data members (which are handled).
\item[Object R-values] There is no treatment of returning class
  r-values from functions, or constructing temporaries by explicitly
  calling constructors inside expressions.  In turn, there is thus no
  treatment of the acceptable elision of construction of some
  temporaries.
\item[Object Sizes and Layouts] The model is almost certainly too
  concrete in its treatment of memory.  Discussion on the
  \texttt{comp.std.c++} newsgroup has convinced me that there is no
  real requirement in the standard for class objects to be laid out in
  any particular way, nor even consistently across instances of the
  same class type.  (The only real requirement is that POD structs
  have their members occur in the order in which they are declared.)
\item[Overloading] Completely omitted. Overloading is arguably a
  static concern (certainly, in contrast with polymorphic dispatch, it
  is handled entirely at compile-time).  Nonetheless it is an
  important part of the language, and gives it much of its ``feel''.
\end{description}

\section{Future}

The next deliverable calls for a treatment of exceptions and
templates.  My notes for deliverable 1 cover how I believe exceptions
will be handled.  Templates remain unclear.  I would also appreciate
advice on which of the wrinkles and omissions above are seen as
significant and worthy of further attention.

\bibliographystyle{plain}
\bibliography{deliverables}

\end{document}

%%% Local Variables:
%%% mode: latex
%%% TeX-master: t
%%% End:
