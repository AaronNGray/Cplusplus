\documentclass[11pt]{article}

\usepackage{charter}
\usepackage{alltt}
\usepackage{url}
\usepackage{proof}
\usepackage{amsfonts}

\newcommand{\cpp}{\mbox{C\hspace{-0.5pt}+\hspace{-2pt}+}}
%\newcommand{\cpp}{C\nolinebreak\hspace{-.05em}\raisebox{.4ex}{\tiny\bf +}\nolinebreak\hspace{-.10em}\raisebox{.4ex}{\tiny\bf +}}

\newcommand{\ruleid}[1]{\textsl{[#1]}}

\newcommand\ckey[1]{\texttt{#1}}

\newcommand\statebrack[1]{\langle #1 \rangle}
\newcommand\exprarrow{\rightarrow_e}
\newcommand\tcexarrow{\exprarrow^*}
\newcommand\expreval[2]{\statebrack{#1} \exprarrow \statebrack{#2}}
\newcommand\tcexeval[2]{\statebrack{#1} \tcexarrow \statebrack{#2}}
\newcommand\stmtarrow{\rightarrow_s}
\newcommand\stmteval[2]{\statebrack{#1} \stmtarrow \statebrack{#2}}
\newcommand\vardarrow{\rightarrow_v}
\newcommand\vardeval[2]{\statebrack{#1}\vardarrow \statebrack{#2}}
\newcommand\hoaretriple[3]{\{#1\}\;#2\;\{#3\}}
\newcommand\hoarequad[4]{\{#1\}\;#2\;\{#3\}\{#4\}}

\newcommand\inferwhere[1]{#1\par\nopagebreak\vskip 1mm\nopagebreak%
                          \noindent\ignorespaces}

\newcommand\okval{\ckey{okVal}}
\newcommand\tyobj[1]{\operatorname{\textbf{obj}}[#1]}

\newcommand\addse{\ckey{add\_se}}
\newcommand\applyse{\ckey{apply\_se}}
\newcommand\clearse{\ckey{clear\_se}}
\newcommand\declvar{\ckey{decl\_var}}
\newcommand\initdeclvar{\ckey{init\_decl\_var}}
\newcommand\installparams{\ckey{inst\_parms}}
\newcommand\markref{\ckey{mark\_ref}}
\newcommand\memtrans{\ckey{mem\_trans}}
\newcommand\removerefs{\ckey{remove\_refs}}
\newcommand\structdeclare{\ckey{struct\_decl}}

\newcommand\cid{{\sf id}}
\newcommand\cgenop{\odot}
\newcommand\Cand{\ckey{\&\&}}
\newcommand\Cor{\ckey{||}}
\newcommand\ccond[3]{#1 \;\ckey{?}\; #2 \;\ckey{:}\; #3}
\newcommand\cassign[2]{#1\mathbin{\ckey{=}}#2}
\newcommand\ccast[2]{\ckey{(}#1\ckey{)}#2}
\newcommand\copassign[3]{#1\mathbin{#2\ckey{=}}#3}
\newcommand\andorsqpt{\widehat{\ckey{\&|}}}
\newcommand\cptr{\ptrtype}
\newcommand\se[2]{\clubsuit(#1,#2)}
\newcommand\rhsop{\diamondsuit}
\newcommand\ass[4]{%
  #2 \mathbin{\raisebox{-1.5mm}{$\stackrel{#4}{#1\ckey{=}}$}} #3}
\newcommand\cif[3]{%
\mbox{\ckey{if} \ckey{(}$#1$\ckey{)} $#2$ \ckey{else} $#3$}}
\newcommand\cloop{\mbox{$O$}}
\newcommand\trap{\mbox{$T$}}

\newcommand\declstmts{\mbox{\sl decls}\;\mbox{\sl stmts}}
\newcommand\cblock[1]{\mbox{\bf \{} #1\mbox{\bf \}}}
\newcommand\cwhile[2]{\mbox{\ckey{while} \ckey{(}$#1$\ckey{)} $#2$}}
\newcommand\cfor[4]{\mbox{\ckey{for ($#1$; $#2$; $#3$) $#4$}}}
\newcommand\cdowhile[2]{\mbox{\ckey{do $#1$ while ($#2$);}}}
\newcommand\tradwhile[2]{\mbox{\bf while}\;#1\;\mbox{\bf do}\;#2}
\newcommand\tradif[3]{\mbox{\bf if}\;#1\;\mbox{\bf then}\;#2\;%
                      \mbox{\bf else}\;#3}

\newcommand\ptrtype[1]{#1\ckey{*}}
\newcommand\arraytype[2]{#1\ckey{[}#2\ckey{]}}
\newcommand\sgndtype[1]{\ckey{signed}\;#1}
\newcommand\usgndtype[1]{\ckey{unsigned}\;#1}
\newcommand\void{\ckey{void}}
\newcommand\longdouble{\ckey{long~double}}
\newcommand\structtype[1]{\ckey{struct}\;#1}
\newcommand\struct{\ckey{struct}}

\newcommand\cnum{\ckey{Num}\,}
\newcommand\cfunref{\ckey{Funref}\,}
\newcommand\cnull{0}
\newcommand\cvar{\ckey{Var}\,}
\newcommand\caddr[1]{\ckey{\&}#1}
\newcommand\cderef{\ckey{*}\,}
\newcommand\cbreak{\ckey{break}}

\newcommand\cholundef{{\cal U}}
\newcommand\cvalue[2]{\underline{(#1,#2)}}
\newcommand\clvalue{\ckey{LV}}
\newcommand\cfvalue{\ckey{FV}}
\newcommand\crvreq{\ckey{RVR}}


\newcommand\retval{\ckey{RetVal}}
\newcommand\stmtval{\ckey{StmtVal}}
\newcommand\breakval{\ckey{BreakVal}}
\newcommand\contval{\ckey{ContVal}}

\newcommand\vardval{\ckey{VarDeclVal}}


\newcommand\wftype[2]{#1\vdash \text{\emph{wf}}(#2)}
\newcommand\etyping[3]{#1\vdash #2 : #3}

\newcommand\ctilde{\ckey{\~{}}}

\newcommand\lvtorv{\leadsto}
\newcommand{\funtype}[2]{#1 \rightarrow #2}
\newcommand{\SOME}[1]{\lfloor #1 \rfloor}

\newcommand{\last}{\textsf{last}}
\newcommand{\hd}{\textsf{hd}}

\newenvironment{proof}{\sl\setlength\parindent{0mm}\addtolength\parskip{2mm}}%
{}

\newenvironment{stdrule}{\begin{center}\begin{minipage}{\textwidth}\begin{alltt}}{\end{alltt}\end{minipage}\end{center}}

% Local Variables:
% mode: latex
% TeX-master: t
% End:


\title{Notes on Deliverable 2 (August 2006)}
\author{Michael Norrish\\{\small \texttt{Michael.Norrish@nicta.com.au}}}
\date{}


\begin{document}
\maketitle

\section{Form}

This deliverable consists of a compressed \texttt{tar}-file, that when
unpacked consists of a directory called \texttt{qinetiq-cpp}, which in
turn contains three directories
\begin{itemize}
\item \texttt{holsrcs}, containing the HOL source files of the
  mechanisation.  These files will build with the version of HOL4
  present in the CVS repository at SourceForge, with timestamp
  \texttt{2006-07-31 00:00Z}.  See Section~\ref{sec:getting-hol}
  below for instructions on how this version of HOL can be retrieved,
  and how the deliverable's HOL source files can then be built and
  checked.
\item \texttt{talks}, containing the \LaTeX{} source and a PDF for the
  talk presented at the DARP meeting in Newcastle in April 2006.  The
  source assumes that the \texttt{latex-beamer} and \texttt{PSTricks}
  packages are available.
\item \texttt{docs}, containing \LaTeX{} sources and a PDF version of
  this document, as well as sources for the notes on Deliverable~1.
\item \texttt{notes}, some C++ source files that illustrate various
  aspects of C++ behaviour.  An accompanying text file explains some
  of the behaviours.
\end{itemize}

\subsection{Building HOL Source-Files}
\label{sec:getting-hol}

\paragraph{Getting HOL From SourceForge}

To get a particular, dated, version of the HOL4 sources from the CVS
repository, one must first issue the command

{\small
\begin{verbatim}
   cvs -d:pserver:anonymous@hol.cvs.sourceforge.net:/cvsroot/hol login
\end{verbatim}
}

When prompted for a password, just press \texttt{Enter} to send a null
response.  The check-out of source code from SourceForge can now
proceed.  The source code fits into 60MB.  Issue the command

{\small
\begin{alltt}
   cvs \textit{repository-spec} co -D \textit{date} hol98
\end{alltt}
}

\noindent where \textit{\ttfamily repository-spec} is the string

{\small
\begin{alltt}
   -d:pserver:anonymous@hol.cvs.sourceforge.net:/cvsroot/hol
\end{alltt}
}

\noindent (also used in the \texttt{login} command), and where
\textit{\ttfamily date} is the desired date, best specified as an
ISO~8601 string enclosed in double-quotes.  For example,
\texttt{"2006-06-30 04:05Z"}.

Once a copy of the sources have been downloaded, further commands can
be used to update this copy to correspond to different dates.  As long
as such commands are issued from within the \texttt{hol98} directory,
the repository specification can be omitted.  The update command is

{\small
\begin{alltt}
   cvs update -d -D \textit{date}
\end{alltt}
}

\paragraph{Installing HOL} Once the sources have been downloaded, the
installation instructions from the page at
\url{http://hol.sourceforge.net} should be followed to build a copy of
HOL.  An installation of the Moscow~ML compiler (v2.01) will also be
required.

\paragraph{Building Deliverable Sources}
When HOL4 has been installed, the \texttt{Holmake} program (found in
the \texttt{hol98/bin} directory) can be run in the \texttt{holsrcs}
directory to create and check the logical theories.

\subsection{Form of the Semantics}

There is a new file in the \texttt{holsrcs} directory,
\texttt{class\_infoScript.sml}.  This contains definitions that
analyse a state's class hierarchy, principally modelled on the
functions defined in
Wasserrab~\emph{et~al}~\cite{wasserrab-nst-OOPSLA06}.

Rules in the dynamic semantics are now being given labels in comments
tagged with the string \texttt{RULE-ID}.  Such rules are referred to
in this document with the following format: \ruleid{sample-rule-id}.

\section{Content}

This deliverable delivers a HOL semantics for a ``\cpp-like''
language.  In accordance with the statement of work, it provides
``basic object-orientation (dynamic dispatch, inheritance) and
reference types''.  Object-orientation is discussed in
Section~\ref{sec:basic-oo} below, while reference types are discussed
in Section~\ref{sec:reftypes}.  Some preliminary work on validation
technology is discussed in Section~\ref{sec:validation}.

\subsection{Basic Object-Orientation}
\label{sec:basic-oo}

The inspiration for this part of the semantics is the article by
Wasserrab~\emph{et al}~\cite{wasserrab-nst-OOPSLA06}, which provides a
detailed model for multiple inheritance.  For the moment, my model
doesn't handle multiple inheritance, but I have nonetheless adopted
much of this article's basic technology as it certainly handles the
single-inheritance situation.

\subsubsection{Class Declaration}
\label{sec:class-declaration}
A class declaration is similar to the C model's declaration of a
\texttt{struct} type.  A class declaration takes two parameters, the
name of the class, and an optional ``class-info''
argument.\footnote{The information argument is optional to allow the
  situation where a forward declaration of a class occurs.}  The
class-info, if present, is a list of fields, coupled with an optional
ancestor class.  The latter allows single-inheritance.  The fields are
of two sorts, ``data'' fields or function definitions.  (A field
declaration of a function, though strictly not data, emulates the
situation where a member function is declared elsewhere.)  Both sorts
of fields are accompanied by a flag indicating whether or not they are
static, and a protection indicator (i.e., \texttt{public},
\texttt{protected}, or \texttt{private}).  For the moment, both
static-ness and protection information is entirely ignored, so that
all members are assumed to be \texttt{public} and non-static.

Member function definitions give the function's name, return-type,
parameter list (types and names), and function body.  This is a close
match for the abstract syntax.  All functions are assumed to be
virtual.

When a class declaration is encountered (only at the top-level;
neither nested or local classes are supported yet), its member
functions are entered into the state's function map.  The same
function map is used for normal (non-class) functions, so there is a
type of ``function-id'', declared
\begin{verbatim}
   fnid = GlobalFn of string | MFn of string => string
\end{verbatim}
meaning that such an identifier can be either global, in which case it
is given by one string (the function's name); or it can be a member
function, in which case it is given by a pair of strings: the class's
name and the function's name.

\subsubsection{Class Values and Dynamic Dispatch}
\label{sec:class-values}
There is no support for classes as r-values in the semantics as yet.
Instead, all expressions of class-type must be l-values.  This
restriction is based on the problems that will arise when multiple
inheritance is implemented.  In particular, with multiple inheritance
in place, it is no longer true that one can extract the byte sequence
for a given l-value by starting at the base address and taking as many
bytes from memory as the size of the type.  In particular, virtual
base classes may be at completely different places in memory, not
necessarily even contiguous with the rest of the object.  (This is
demonstrated for the \texttt{g++} compiler by the little program in
\texttt{notes/mult-inheritance-layout.cpp}.)

Absence of support for r-values means that, in this model, one can not
assign classes, pass them as parameters, or return them from
functions.  However, because references are supported (see below),
much idiomatic \cpp{} is still covered by the model.

The presence of classes means that the model's presentation of
l-values changes from the way it was in the original C model.  In
particular, an l-value that is statically typed as a base class needs
to know dynamically that it is really of a derived type.  This
information is traditionally recorded in \texttt{vtable} fields.
Following Wasserrab~\emph{et al}, my model instead records an
additional path accompanying every l-value.  This path is a list of
strings, listing the path through the ancestry-tree that leads from
the most-derived class to the current class type.

Consider, for example, the code in Figure~\ref{fig:oo-example}.  The
body of function \texttt{g} constructs the l-value \texttt{*b} when it
calls \texttt{f}.  As in the C model, this l-value will be associated
with some address, and the static type, which is \texttt{Class~B}.
But the additional path information that accompanies the l-value will
be the list \texttt{[D2,D1,B]}.  The last element of such paths is always
the static class type.  The prefix of the list makes it clear that
this is a \texttt{B} l-value that is dynamically a \texttt{D2}.  And
when field selection occurs, the \texttt{f} chosen will be
\texttt{D1}'s.

\begin{figure}[hbtp]
\begin{verbatim}
   class B {
     public: virtual int f() { return 3; }
   };

   class D1 : public B {
     public: int f() { return 4; }
   };

   class D2 : public D1 { };

   int g(class B *b) { return b->f(); }

   int main()
   {
     D2 d;
     return g(&d);
   }
\end{verbatim}
\caption{C++ code demonstrating OO-polymorphism.  The program will
  return 4. Though it appears as if \texttt{B}'s function \texttt{f}
  is called, the version of \texttt{f} called will actually be that
  attached to \texttt{D1}.}
\label{fig:oo-example}
\end{figure}

The rule encoding all this is \ruleid{function-member-select}:
\[
\infer{
  \begin{array}{l}
    \statebrack{\textsf{Ex}((\clvalue(a,\texttt{Class}\;C,p))\;\mathbf{.}\;
      \mathit{fld},\mathit{se}),\sigma} \rightarrow \\
    \qquad\left\langle\textsf{Ex}\left(\cfvalue\left(
        \begin{array}{l}
          \texttt{MFn}\;(\last(p'))\;\mathit{fld},\\
          (\funtype{\mathit{argtys}}{\mathit{ret}}),\\
          \SOME{(\clvalue\;a\;(\texttt{Class}\;(\last(p')))\;p')}
        \end{array}\right),
      \mathit{se}\right),\sigma\right\rangle
  \end{array}}{
  \sigma \vdash \hd(p) \textrm{ has least } \mathit{fld} :
  \funtype{\mathit{argtys}}{\mathit{ret}} \textrm{ via } p'}
\]

In the figure, the expression of interest is \texttt{b->f()}, which is
syntactic sugar for \texttt{((*b).f)()}.  (Note how the member
selection is syntactically subordinate to the function application.)
The \ruleid{function-member-select} rule governs the evaluation of
\texttt{(*b).f} once the left-hand-side (\texttt{*b}) has evaluated to
an l-value.\footnote{Even when the model allows for class r-values,
  they will be given a memory location (and thus, a life-time).  This
  will enable them to also be l-values.  In essence, it is not
  possible to create an object of any sort without giving it a
  location.  Contrast numbers, which can be ``created'', and not given
  a memory location, simply by writing them down.}

In our model, the l-value's address will be the same as the address of
its most-derived class.  In other words, the $a$ of the rule will be
the same as the address of the object \texttt{d}.  This is not what
happens in typical compilers, which will actually make the pointer to
the \texttt{B} sub-class point at the address of that sub-object's
fields in the wider object's memory layout.  Then, the fact that the
most-derived class is a \texttt{D2} is implicitly recorded in the
\texttt{vtable}, which will contain a pointer to \texttt{D1::f}.

In the rule above, the variable $C$ containing the name of the static
class will be \texttt{B}, and $p$, the path variable, will be
\texttt{[D2,D1,B]}.  The $\mathit{fld}$ variable will be \texttt{f}.
Then, the hypothesis will check the class hierarchy to determine where
an \texttt{f} can be found, starting at the head of the path, i.e., at
\texttt{D2}.  This will reveal that there is an \texttt{f} at path
\texttt{[D2,D1]}.  This will be the value for $p'$.

The RHS of the reduction arrow $\exprarrow$ features a new sort of
expression constructor, \cfvalue{}.  This is a ``function l-value'', a
designation of a particular function within the program.  The first
argument is the function identifier, as described earlier in
Section~\ref{sec:class-declaration}.  In this case, the function
identifier points at a class's member function given by two strings:
the name of the class where the function body is declared, and the
field-name.  In our example, the function identifier will be
\texttt{MFn~D1~f}, or simply \texttt{D1::f} in more traditional \cpp{}
notation.  The second argument of the \cfvalue{} constructor is the
type of the function.  The third argument is an optional expression
containing the l-value for the object associated with a member
function call.  (The use of $\SOME{\;}$ marks that this is an optional
value.)  This third argument is not present if the function is a
normal one (i.e., not a class-member).

There are two rules for performing function applications, but the only
difference between them is that the member function version sets up
the value of the \texttt{this} expression form.  Otherwise, both rules
extract the body for the given function-identifier, and set up
parameters in the same way.

\paragraph{Field Selection}
Field selection is also based on the notion of being able to find the
most-derived declaration of the given field in the ancestor hierarchy.
There is no need to worry about adjusting \texttt{this} pointers, or
performing analyses with dynamic types as field selections are always
done with respect to a class's static type.  However, there is an
additional complexity, stemming from the need to give addresses to
selected fields, so that they can become well-formed l-values.  In
turn, this relies on knowing how a class is laid out in memory.

The standard \emph{does} require that fields belonging to a particular
class type be laid out in the order in which they appear (as long as
they have the same access-specifiers, but I'm not modelling those just
yet).  But there is no specification of how base sub-objects are laid
out.  (Recall, moreover, that in the presence of a virtual base-class,
an object that is not most-derived may be split over distinct parts of
memory.)

With just single inheritance to model, I use an under-specified
function \texttt{class\_layout} which takes a set of sub-object paths,
and returns them in a list.  This then drives the calculation of field
offsets and class sizes.

\paragraph{What of vtables?}
The use of paths does away with the need for vtables.  On the other
hand, we wouldn't want to specify the model in such a way as to
preclude this perfectly reasonable implementation strategy.  In
particular, vtables will be catered for just as in the standard, by
maintaining that it is only in POD (``plain old data'') types where
one can rely on the address of the first field being the same as the
address of the containing \texttt{struct}.  The calculation of sizes
will also be under-specified to allow for the presence of extra,
user-invisible data at the start of a class.  (This latter hasn't been
done yet.)

\paragraph{An Axiom}
The mechanisation includes an axiomatisation stating that two
otherwise unspecified constants, \texttt{ptr\_encode} and
\texttt{ptr\_decode} are mutual inverses.  These partial functions map
between the dynamic arguments that constitute l-values (address and
path), and the lists of bytes that make up pointers.

\subsection{Reference Types}
\label{sec:reftypes}

Reference types pose problems in the contexts where they are
distinctive:
\begin{itemize}
\item passed as parameters to functions;
\item returned from functions;
\item when they are initialised.
\end{itemize}

Otherwise, it is obvious how the existing semantics should treat
references: they are l-values.  References are only created from
l-values, and in the reverse direction, values will turn into r-values
as required.  They will do the right things when of \texttt{class}
types because they will have the Wasserrab style paths attached.  In
other words, a reference to a base class may actually be an l-value
referring to a derived class.

The remaining sub-sections explain what happens in the three
interesting situations.

\paragraph{Omissions}
\begin{itemize}
\item The semantics does not handle references to functions.
  Supporting these will require more rules in the dynamic semantics,
  but these rules will be directly analogous to the rules presented
  below: wherever an \clvalue{} constructor (which is for l-values of
  object type) appears, there will need to also be a rule for
  \cfvalue{}, which is for function values.
\item The semantics does not handle initialisation of references
  belonging to classes because these have to happen in the
  \emph{mem-initialisors} of a class's constructor function(s), and
  constructor functions are not yet modelled.
\item The semantics doesn't handle initialisation of \ckey{const}
  references from r-values because it does not yet handle
  \ckey{const}.
\end{itemize}


\subsubsection{References as Parameters}

When a formal parameter is of reference type, the actual parameter
needs to stay an l-value rather than reduce to an r-value.  When the
called function's environment is being established in its local stack
frame, the binding for the formal name can be directly to the actual
l-value's address and type.  In other words, nothing is allocated in
memory to represent the reference.  This is not very likely in an
actual environment, which will probably have an address in memory
somewhere.  (The only way to detect this (and this would require
undefined behaviour) would be know where local variables were
allocated, and to scan this area byte-by-byte, presumably starting at
the address of some other, non-reference, local parameter.)

This requires a change to the existing semantics, to remove function
arguments from the $\cal L$ (l-value) context.  In order to allow some
function arguments to decay into r-values, a new rule
\ruleid{function-application-lval2rval} is introduced:
\[
\infer{
  \statebrack{\textsf{Ex}(f(\mathit{pfx} \frown e_0 ::
    \mathit{sfx}),
    \mathit{se}_0),s_0}
  \rightarrow
  \statebrack{\textsf{Ex}(f(\mathit{pfx}\frown e ::
    \mathit{sfx}),\mathit{se}),
    s}}{%
  \statebrack{\mathit{se}_0,e_0,s_0}\lvtorv\statebrack{\mathit{se},e,s} &
  f\mbox{ expects an r-value at position }|\mathit{pfx}|}
\]
The variables $\mathit{pfx}$ and $\mathit{sfx}$ are lists of
expressions corresponding to the other actual parameters being passed
to the function $f$.  The $\frown$ operator is list concatenation, and
$::$ is ``cons'', taking an element and putting onto the front of
another list ($::$ binds tighter than $\frown$).

The rule that determines that a function application's sequence point
has been reached (\ruleid{function-call-sqpt}) must also change.
Previously, this rule checked that the function and all of the
arguments had been fully evaluated, and being ``fully evaluated''
meant ``had been evaluated to a value (consisting of a list of
bytes)''.  The new rule checks that every parameter is either a
byte-list value (when the function doesn't expect a reference type),
or an l-value.

\subsubsection{References Returned from Functions}

This requires a number of changes to the semantics.  First, there
shouldn't be an \crvreq{} tag wrapped around the expressions that are
arguments to \texttt{return} statements.  Secondly, the continuation
that accompanies all statement evaluations (as of Deliverable-1),
needs to be adjusted so that it is one of two possibilities, a
continuation accepting a value, or a continuation accepting an
l-value.  The decision as to which sort of continuation it will be is
set up when a function call is made, because the desired return type
of the function is known at this point.

The new type of continuations is an algebraic type with two
constructors:
\begin{verbatim}
   conttype =
       RVC of (byte list -> CPP_Type -> CExpr)
     | LVC of (num -> CPP_Type -> string list -> CExpr)
\end{verbatim}
The rules for \texttt{return} are able to examine the type of
continuation directly.  So that \ruleid{return-lval2rval} is
\[
\infer{\statebrack{\textsf{St}(\ckey{return}\;\textsf{Ex}(e_0,\mathit{se}_0)\ckey{;},\;\ckey{RVC}(c)),s_0}\stmtarrow
  \statebrack{\textsf{St}(\ckey{return}\;\textsf{Ex}(e,\mathit{se})\ckey{;},\;\ckey{RVC}(c)),s}}{
  \statebrack{\mathit{se}_0,e_0,s_0}\lvtorv\statebrack{\mathit{se},e,s}}
\]
Every statement is accompanied by a continuation, the second argument
of the \textsf{St} tag. Also note that the first argument of a
\ckey{return} might itself be another statement.  If the original
expression contained a function call, the body of the called function,
a statement, would eventually become the top of the expression.  In
this rule, the use of the inner \textsf{Ex} tag precludes this
possibility.

There are two rules allowing a \ckey{return} statement to pass its
expression-value to the continuation.  The rule for returning normal
values is \ruleid{return-rvalue}:
\[
\infer{%
  \begin{array}[t]{l}
    \statebrack{\textsf{St}(\ckey{return}\;\textsf{Ex}(\cvalue{m}{\tau},\mathit{se})\ckey{;},\;\ckey{RVC}(c)),
      \;\sigma}%
    \;\;\rightarrow\\
    \qquad\left\langle\!\!\begin{array}{l}
      \textsf{Ex}(c\;m\;\tau,\mathit{empty\_se}),\\
      \sigma[\textsf{stack} := \mathit{stack}', \textsf{this} :=
      \mathit{this}, \Gamma := g, \Sigma := s, A
      := a]\end{array}\right\rangle
  \end{array}}{
  \mathit{se}\mbox{ is empty of side effects} & \textsf{stack}_\sigma = (s,g,a,\mathit{this})
  :: \mathit{stack}'}
\]
The \textsf{stack} and \textsf{this} fields of the state are not in
the original thesis semantics.  The \textsf{stack} field records the
stack of environment information.  The environment information
includes maps for variable typing ($\Gamma$), address information for
variables ($A$), class information ($\Sigma$) and the current
(l-)value for \ckey{this}, if there is one.   Use of the
\textsf{stack} is a necessary part of moving to a small-step semantics
for statements.

The rule for returning a reference \ruleid{return-lvalue} is similar:
\[
\infer{%
  \begin{array}[t]{l}
    \statebrack{\textsf{St}(\ckey{return}\;\textsf{Ex}(\clvalue(\mathit{ad},\tau,p),
      \mathit{se})\ckey{;},\;\ckey{LVC}(c)),
      \;\sigma}%
    \;\;\rightarrow\\
    \qquad\left\langle\!\!\begin{array}{l}
      \textsf{Ex}(c\;\mathit{ad}\;\tau\;p,\mathit{empty\_se}),\\
      \sigma[\textsf{stack} := \mathit{stack}', \textsf{this} :=
      \mathit{this}, \Gamma := g, \Sigma := s, A
      := a]\end{array}\right\rangle
  \end{array}}{
  \mathit{se}\mbox{ is empty of side effects} & \textsf{stack}_\sigma = (s,g,a,\mathit{this})
  :: \mathit{stack}'}
\]

\subsubsection{Declaring References}

When a reference is declared, it must also be initialised, and the
initialising expression must be an l-value.  But normal
initialisations need to be able to turn l-values into r-values so
there is an ``lvalue-to-rvalue'' rule for variable declarations that
are accompanied by initialisations.  This rule is
\ruleid{block-vdecinit-lval2rval}:
\[
\infer{\begin{array}{l}
    \statebrack{\textsf{St}(\mathbf{\{}
  (\textsf{VDecInit}\;\tau\;\mathit{name}\;(\textsf{Ex}(e_0,\mathit{se}_0))
  :: \mathit{vds})\;\mathit{sts} \mathbf{\}}, c),\sigma_0}
\rightarrow\\[2mm]
\qquad\statebrack{\textsf{St}(\mathbf{\{}
  (\textsf{VDecInit}\;\tau\;\mathit{name}\;(\textsf{Ex}(e,\mathit{se}))
  :: \mathit{vds})\;\mathit{sts} \mathbf{\}}, c)),\sigma}\end{array}}{
\tau \mbox{ not a reference type} &
\statebrack{\mathit{se}_0,e_0,\sigma_0}\lvtorv
  \statebrack{\mathit{se},e,\sigma}}
\]
where $\textsf{VDecInit}\;\tau\;\mathit{name}\;e$ is a value in the
abstract syntax type corresponding to the declaration of variable
$\mathit{name}$ with type $\tau$, to be initialised with expression
$e$.  The braces in the syntax above indicate that this happens in the
context of a statement-block.  (There are corresponding rules for
variable declarations and initialisations at the top-level of a
translation unit.)

A variable declaration of reference type is ready to ``fire'' when its
initialising expression has reduced to an l-value, and when there are
no remaining side effects in the side effect record.  At this point,
the variable address map in the state is updated to point at the
l-value's address, and the type map is also updated to make the type
of the name the same as the type of the l-value.


\subsection{Validation}
\label{sec:validation}

Deliverable-2 includes a directory \texttt{holsrcs/testfiles}.  In
this directory there is some preliminary work towards the creation of
a symbolic evaluator, for demonstrating that programs in the model can
behave as one might expect.  This work builds on the ideas
in~\cite{netsem:popl2006}, allowing symbolic exploration of a semantic
definition that features non-deterministic branching.  For the moment,
the code only handles a sequence of external declarations not
requiring any expression evaluation.

In order to make using this tool even easier, I will need to develop
some sort of parser for C++ source code as having to write out
abstract syntax trees by hand is a major annoyance.  It's possible
that the front-end of \texttt{g++} might be usable in this regard.

\section{Future}

The next deliverable will include a treatment of multiple inheritance,
\texttt{const} types and object lifetimes.  Multiple inheritance
should be straight-forward, given the Wasserrab's model.  The
treatment of \texttt{const} should also be fairly easy.  Object
lifetimes will require a treatment of class constructors and
destructors at the very least, and could well be moderately
complicated, particularly if the standard's level of
under-specification is to be modelled.

If destructors are to be called before a certain deadline is reached,
this might best be modelled in a way similar to the treatment of side
effects.  Object destructions can be accumulated in some sort of state
field (a set, say), and non-deterministically pulled from this as
evaluation progresses.  When a deadline is reached, there will be a
side-condition on the relevant rule requiring that the state field be
empty.

\bibliographystyle{plain}
\bibliography{deliverables}

\end{document}


%%% Local Variables:
%%% mode: latex
%%% TeX-master: t
%%% End:

